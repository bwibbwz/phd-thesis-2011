\chapter{Coupled Hydrogen Defects in Perovskites}
\label{chap:perovskites}
\section{Introduction}
\label{sec:perovskites-introduction}

As a DFT test system for the ridge detection method, a system that was the subject of a recent Ph.D. thesis~\cite{nicolai-2010} in Tejs VEgge's group was chosen.
The coupled hydrogen defect in a \ce{SrTiO3} perovskite structure.~\cite{double-defect-2011}
Usage of such systems can be varied, \expand
However, in the context of this thesis it was chosen as a real world example with noisy forces to briefly test the ridge detection method.

It is of particular interest to see if the diffusional process is coupled.
This had been shown in \cite{double-defect-2011} \expand





The results of Nicolai were somewhat confirmed by performing Dimer calculations which showed that the transitions of interest (low energy barriers) were, in fact, mostly those where the \ce{H} atoms would stick close to each other.
However, medium range \ce{H}-\ce{H} interaction was impossible to get reliable results for due to the size of the calculational cell, but numerous such processes were found.

\section{DFT Ridge Calculations}
In an effort to test the ridge calculations on DFT systems, the di-proton interstitial defect in \ce{SrTiO3} perovskites~\citemiss was briefly looked at.

Picking the most relevant \sap{1}s as endpoints a few ridge calculations were attempted.
Well behaved cases converged quickly but some would get trapped near minima and spend hundreds of iterations there until the calculations were aborted.
It is unlikely that this minima-trapping behaviour is an artefact of the method itself as such things did not happen in the previous test cases.\footnote{Or did they in the difficult 3-atom vs. 2-atom concerted cases?}


