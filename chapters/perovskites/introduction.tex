\section{Introduction}
\label{sec:perovskites-introduction}

As a DFT test case for the ridge mapping method, the \ce{SrTiO3} perovskite system was chosen, partly due to general interest in its properties and partly due to familiarity, as it was the one of the subject of a recent Ph.D. thesis~\cite{nicolai-2010}.

Perovskites have potential in a number of fields and active research is, for example, being done on proton transport processes~\cite{perovskites-hydrogen-diffusion-2007, perovskites-proton-transport-2008} for use in solid oxide fuel cell membranes, that allows hydrogen to pass\cite{perovskites-in-fuel-cells-1981}.
Other than this, perovskites are used in a number of applications, such as electrodes, semiconductors and solar cells.

The system under study in this chapter is a double hydrogen defect in \ce{SrTiO3}.~\cite{double-defect-2011}.
It has been found that the double defect is more stable than an isolated one and that coupled motion of the pair is marginally faster than that of a single atom.
In this chapter the system serves as a complex real-world application test of the ridge mapping method.
Briefly touching on the initial efforts to apply the ridge mapping method to noisy (DFT) forces, neither a comprehensive study is presented nor a detailed look at the system in question.

%Nevertheless, it warrants inclusion in this thesis as the application of such methods to DFT systems is an important one.
As for a more detailed information to perovskites, the reader must refer to the citations in this intoduction.

%a system that was the subject of a recent Ph.D. thesis~\cite{nicolai-2010} in Tejs Vegge's group was chosen due to familiarity.

%As a DFT test case for the ridge detection method, a system that was the subject of a recent Ph.D. thesis~\cite{nicolai-2010} in Tejs Vegge's group was chosen due to familiarity.
%The coupled hydrogen defect in a \ce{SrTiO3} perovskite structure.~\cite{double-defect-2011}
%Usage of such systems can be varied, \expand
%However, in the context of this thesis it was chosen as a real world example with noisy forces to briefly test the ridge detection method.

%It is of particular interest to see if the diffusional process is coupled.
%This had been shown in \cite{double-defect-2011} \expand

%Due to time constrictions this chapter can only briefly touch on the subject of ridge calculations with noisy (DFT) forces.
%Neither could the study be comprehensive nor detailed.
%Much rather, a series of problems were encountered and analysed within a very short period of time.

