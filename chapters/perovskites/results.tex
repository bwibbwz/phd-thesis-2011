\section{Results}
\label{sec:perovskites-results}
\subsection{Confirming Previous Results}
The results of \cite{double-defect-2011} showed that the hydrogen atoms would move in individual steps while staying close to each other.
Two such mechanisms were discussed and both consisted of multiple iterations of two types of events.
Either the transitioning hydrogen atom would "jump" from one oxygen atom to the next near a titanium atom or it would "rotate" past a strontium atom while remaining near the same oxygen atom.
Both hydrogen atoms would perform these steps --- often in alternating order --- while remaining in close proximity of each other.

Dimer \sap{1} searches were conducted, starting from the various minima suggested in \cite{double-defect-2011}.
Essentially, confirming the previous results, the low energy \sap{1}s were all events of the types described above.
Only a handful of truly concerted events, where both hydrogen atoms would transition simultaneously, were detected but they were all higher in energy.

Some \sap{1}s were found at a slightly longer range, e.g. where the hydrogen atoms would be separated by a strontium atom.
In the calculational cell used this is effectively half of its length, bringing into question any discussion on the energies as periodic effects could both over- and underestimate the stabilisation by the neighbouring hydrogen atom.
Further investigation of these might prove interesting but a larger calculational cell is required.

\subsection{DFT Ridge Calculations}
The same parameters were used for the dimer part of the ridge calculations as were used for the dimer searches above.

A few ridge calculations were started with some of the \sap{1}s found above as endpoints.
The general results can be split into three categories.
\bit
\item Successful ridge calculations, converged from a linear interpolation to a ridge in under 300 iterations.
\item Unsuccessful calculations where the path would spent most of its time as an inverted barrier near the minimum energy path.
\item Calculations where a soft eigenmode was followed resulting in a non-converging calculation where the lattice would distort heavily.
\eit

\subsubsection{Successful DFT Ridge Calculations}
\begin{figure}[h]
\begin{center}
  \subfigure[The configuration of the A-jump \sap{1}. One titanium atom is semi-transparent to allow viewing of hydrogen B]{
    \includegraphics[width=0.45\linewidth]{semi-combined}
    \label{fig:semi-combined}
    }
  \subfigure[Energy profile of the ridge]{
    \includegraphics[width=0.45\linewidth]{semi-ridge}
    \label{fig:semi-ridge}
    }
    \parbox{0.85\linewidth}{
      \caption{A successful DFT ridge calculation between a jump \sap{1} for hydrogen atom A and rotation of hydrogen atom B.
The hydrogen atoms are white, the titanium atoms are grey, the strontium atoms are green and the oxygen atoms are red.
      }
      \label{fig:semi-results}
    }
\end{center}
\end{figure}
The most interesting of the successful ridge calculations was one where the ridge near a jump-type \sap{1}s (\fref{fig:semi-combined}) consisted of a rather flat energy profile (\fref{fig:semi-ridge}) with lattice rearrangement before a sharp rise to reach a $0.08\unit{eV}$ \sap{2} on its way to a rotation-type \sap{1} for the other hydrogen atom.
Both the flat energy profile and the low \sap{2} are clear warning signs, that the HTST rate could be improved.
\begin{figure}[h]
\begin{center}
  \subfigure[The configuration of the A-jump/B-rotate \sap{2}. One strontium atom removed and one oxygen atom is semi-transparent to allow viewing of hydrogen B]{
    \includegraphics[width=0.45\linewidth]{jump-combined}
    \label{fig:jump-combined}
    }
  \subfigure[Energy profile of the ridge]{
    \includegraphics[width=0.45\linewidth]{jump-ridge}
    \label{fig:jump-ridge}
    }
    \parbox{0.85\linewidth}{
      \caption{A successful DFT ridge calculation between a jump \sap{1} for hydrogen atom A and rotation of hydrogen atom B.
The hydrogen atoms are white, the titanium atoms are grey, the strontium atoms are green and the oxygen atoms are red.
      }
      \label{fig:jump-results}
    }
\end{center}
\end{figure}
A second successful calculation of the ridge near a different jump-type \sap{1} revealed one other low energy \sap{2} of $0.08\unit{eV}$.

From an informal visual inspection of the successful ridge calculations, the initial paths never lay close to a minimum.
However, it was not possible to make the opposite claim for the failed calculations.

\subsubsection{Basin Trapped Paths}
Many of the calculations resulted in paths that lay near the MEP and minimum.
Those calculations that did not reach the ridge, generally lay near the minimum energy path and were not able to rise out of the basin.
The eigenvalue estimate would remain negative for most of the images.
It is unlikely that this minima-trapping behaviour is an artefact of the method itself as such things did not happen in the previous test cases.

\subsubsection{Soft Eigenmodes}
Latching on to a soft eigenmode is a concern that is always possible with dimer-type algorithms.
However, this was not a significant problem in the previous tests.
In this particular case, the climbing image had been turned on very early in order to avoid the basin trapping problem.
After the climbing image is turned with a soft eigenmode, the calculation is unlikely to yield an interesting result.
This problem raises questions about the initial eigenmodes.
If a random initial eigenmode is used (as was the case in this calculation), is it likely that neighbouring eigenmodes will not be similar (i.e. find the same ridge).
It is of course possible to initialise the eigenmodes in a linear fashion, using the unstable modes of the end points or implement some constraint to avoid neighbouring eigenmodes from from being dissimilar.
In any case, this warrants further consideration.

