\chapter{Introduction}
\label{chap:introduction}

\bit
%\item Stationary Points in general
%\item Finding extrema
%\item Other sorts of stationary points
%\item Minima in theoretical chemistry
%\item Saddle points in Chemistry (reaction rates)
\item Why do we want to find higher order saddle points in chemistry.
\item Energy storage, conversion and mobility (Hydrogen as an example)
\item The importance of assessing reaction rates (Kinetics is a very important aspect for the above)
\item General problems with getting reaction rates
\item Annoying landscapes
\item Beyond harmonicity
\item The methods for finding stationary points presented here are inspired by chemistry but are applicable to any function.
\item The thesis is heavily biased towards chemistry
\eit

The border between mathematics, physics and chemistry is often times fuzzy and the methods developed for one may very well benefit the others (for example \expand \citemiss).
Common to all, is some form of functional analysis where the study of stationary points\footnote{Points where the first derivative is zero.} is generally of great interest.
For systems of limited dimensionality, finding stationary points becomes no more complex than solving a root problem for the function's derivative.
However, once systems grow beyond a few dimensions, solving the root problem becomes infeasible.
Furthermore, the lack of a global analytical function \tred{(Is analytical correct in this context?)} makes it impossible to simply solve for the roots.
This is, generally, the case in atomistic simulations, which is the main subject of this thesis.
While the methods discussed herewithin are developed in this field, applying them to purely mathmatical problems and within other fields is mearly a question of implementation, thus they are presented as general as possible in order to appeal to wider audiences.

Locating extrema using local information, i.e. the gradient, is a well known technique\citemiss, whereas the gradient is simply followed in the appropriate direction (positive for maxima and negative for minima) until it vanishes.
Finding other stationary points is less obvious and classifying them depends on the second order derivatives (or even higher in special cases\footnote{e.g. the Monkey Saddle \url{http://mathworld.wolfram.com/MonkeySaddle.html}})
Finding such points without explicitly evaluating higher order derivatives is an important subject within atomic simulations, since their calculation is is very costly when using non-analytical \tred{(Is QM technically non-analytical and what is the exact definition of analytical?)} models to describe the atomic interaction.

\subsubsection{Significance of Stationary Points in Atomic Simulations}
The potential energy of atomic systems can be evaluated as a function of the atomic coordinates.~\cite{born-oppenheimer-1927, schrodinger-equation-1926, kohn-1999}
The interesting features of such a mapping are generally considered to be the minima which represent stable configurations of the atoms.
The transition between minima is then considered to be a chemical reaction.
These can range, for example, from rotations of specific groups with minimal impact on the properties of the material (\fref{chap:borohydrides}) and motion of water molecules in liquid water~\citemiss to adsorption~\citemiss and the breaking of strong chemical bonds~\citemiss.
In understanding the transition in question the path of least resistance is an essential concept, since the highest point along it defines the potential energy that must be overcome.
This point is a non-extremal stationary point and information about the frequency of a given transition (its reaction rate) can be inferred from from it and its surroundings, which is of central importance in chemistry.\cite{htst-wert-1949, htst-vineyard-1957}


\subsubsection{Old Methods Introduction}
The focus of this thesis is the application of theoretical methods to calculational reaction chemistry.
The underlying methods that allow this are numerous and their discovery span ages in time, from Newton's equations of motion~\cite{newton-latin} up to modern day methods for solving Sch\"odinger's equation~\cite{schrodinger-equation-1926} for quantum systems\cite{hohenberg-kohn-1964, gpaw-review-2010}, from the discovery of calculus\citemiss\footnote{see Leibnitz: \url{http://books.google.dk/books?id=UdGBy8iLpocC&pg=PA46&redir_esc=y\#v=onepage&q&f=false}} and vector math~\citemiss to modern methods for solving eigenvalue problems\citemiss.
%This chapter will give a brief look at the main methods employed with special emphasis on those that constitute the basis of the novel method presented in \fref{chap:erm}.

Simulations of atomic systems are an integral part of modern atomic scale research, whether it be to investigate specific electronic structure properties\citemiss, performing atomic dynamics\citemiss or investigating macroscopic properties\citemiss.
Simulations are commonly employed in the analysis of interesting material properties (see for example catalytic properties\citemiss), the screening of candidate materials --- to eliminate the need to synthesize as many unsuccesful materials --- for various processes (see for example hydrogen storage\citemiss) and further collaboration and comparison with experiments (see for example \citemiss).

Reaction chemistry happens on a timescale of microseconds which is, essentially, eternity when seen from the vantage point of the fast moving vibrations, which happen on a timescale of femtoseconds.
Briding this gap is an important research subject which remains open, despite noble efforts that have moved the field a long way\citemiss.

%Dating back to well before modern large scale computer clusters, synamical studies were even performed using tools no more complicated than balls, sticks, pens and paper~\cite{old-simulations-bernal-1962} ATH: Vegetable Staticks (Stephen Hales, 1727)


%\tblue{The order of the topics for discussion in this chapter is still a bit off.
%Born-Oppenheimer is needed before TST but it should be introduced at the same (similar) time as the Schr\"odinger equation which in turn should accompany the DFT section (close to which the potential function section should lie).}
%\bit
%\item Perhaps use the Newton/Schr\"odinger figure I made a bit back (modified if needed)
%\item General on atomistic simulations (seguing to potential functions and DFT)
%\item Why statistical methods
%\item Why path techniques (introduction to be expanded on in the TST section)
%\item It is now "possible" to do long timescale MD simlations but statistical methods are still more suited to find "all" the processes (chat with Elvar on MD)
%\item (Why these methods but not some other?)
%\item Methods that or of particular interest get better coverage.
%
%\item Would be nice to have a figure of each of the main guys for each methodology
%\bit
%\item Born, Oppenheimer
%\item Hesse
%\item Arrhenius, Kramer
%\item Hannes, Graeme
%\item Hohnberg, Kohn
%\eit
%\eit


%\section{Notation}
%\bit
%\item Nuclei positions: $\vR$
%\item Electron positions: $\vr$
%\item R-basis for formulas
%\item The PES: $E(\vR)$
%\eit

\subsubsection{Chapter Outline}
\label{sec:chapters}

\fref{chap:methods}: Methods
\fref{chap:borohydrides}: Metal Borohydrides ...
\fref{chap:erm}: Beyond Harmonicity
\fref{chap:al}: Self-Diffusion of Aluminium ...
\fref{chap:perovskites}: Coupled Hydrogen Defects ...


