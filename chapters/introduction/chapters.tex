\section{Chapter Outline}
\label{sec:chapters}

Perhaps splitting the thesis into parts is worth it, Introduction, Methods, Discussion and Papers

The papers will not be retold, but rather summarised and, where applicable, expanded on, in their respective chapters.

\subsection*{Part \ref{part:theory}}
The methods and concepts that are important to the work presented in the thesis are discussed.
This part is not intended as an exhaustive resource for the methods in question but more as an introduction, sufficient to understand the thesis.
For more in-detail discussion of individual methods, refer to their respective citations.
Basic understanding of vector math and calculus is assumed.

\subsubsection{\Fref{chap:pes}}
\placeholder

\subsubsection{\Fref{chap:tst}}
\placeholder

\subsubsection{\Fref{chap:saddle-point-methods}}
\placeholder

\subsection*{Part \ref{part:thesis}}
\subsubsection{\Fref{chap:borohydrides}}
A brief overview of the theoretical aspects of papers \ref{pap:calcium} and \ref{pap:magnesium} as well as the nature of the collaboration between theory and experiments in the context of limited availablity of experimental results.

\subsubsection{\Fref{chap:erm}}
The methodology of paper \ref{pap:second-order} is presented in a more general context than the paper.
A more detailed look at numerical instabilities in ridge calculations is presented along with an outlook as to minimising said instabilities without sacrificing the accuracy of the ridge determination.

\subsubsection{\Fref{chap:al}}
A look at the self diffusion of \ce{Al} on the \ce{Al(100)} surface is summarised from paper \ref{pap:second-order}.
The results mainly focus on the failures of the harmonic approximation to transition state theory and how to correct for the errors encountered.

\subsubsection{\Fref{chap:perovskites}}
Coupled Hydrogen Defects \expand

