\section{Chapter Outline}
\label{sec:chapters}

The publications (part \ref{part:papers}) will not be retold, but rather summarised and, where applicable, expanded on, in their respective chapters.
Reading the thesis first is recommended, then reading the papers for further detail and context.

\subsubsection{Part \ref{part:introduction} : \nameref{part:introduction}}
A general introduction and a guide to the content of the thesis.

\subsubsection{Part \ref{part:theory} : \nameref{part:theory}}
The methods and concepts that are important to the work presented in the thesis are discussed.
This part is not intended as an exhaustive resource for the methods in question but more as an introduction, sufficient to understand the thesis.
For more in-detail discussion of individual methods, refer to their respective citations.
Basic understanding of vector mathematics, calculus and basic physics is assumed.

\paragraph{\Nref{chap:pes}}
Dynamics of atomic scale systems are discussed.
The concept of a potential energy surface and its creation and interesting features are addressed.

\paragraph{\Nref{chap:tst}}
Reaction rates in theoretical chemistry are non-trivial.
In this chapter, the transition state theory, for calculating reaction rates, is discussed along with its harmonic approximation.

\paragraph{\Nref{chap:saddle-point-methods}}
Methods for locating saddle points are discussed in detail.

\subsubsection{Part \ref{part:thesis} : \nameref{part:thesis}}
The original work of the thesis is presented.

\paragraph{\Nref{chap:borohydrides}}
A brief overview of the theoretical aspects of papers \ref{pap:calcium} and \ref{pap:magnesium} as well as the nature of the collaboration between theory and experiments in the context of limited availability of experimental results.

\paragraph{\Nref{chap:erm}}
The methodology of paper \ref{pap:second-order} is presented in a more general context than in the paper.
A more detailed look at numerical instabilities in ridge calculations is presented along with an outlook as to minimising said instabilities without sacrificing the accuracy of the ridge determination.

\paragraph{\Nref{chap:al}}
A look at the self diffusion of \ce{Al} on the \ce{Al}(100) surface is summarised from paper \ref{pap:second-order}.
The results mainly focus on the failures of the harmonic approximation to transition state theory and how to correct for the errors encountered.

\paragraph{\Nref{chap:perovskites}}
A brief look at DFT ridge calculations of Coupled Hydrogen Defects in a \ce{SrTiO3} perovskite system.
The calculations were only partially successful \expand

\subsubsection{Part \ref{part:papers} : \nameref{part:papers}}
The publications relevant for this thesis.
