\section{Locating Minima}
\label{sec:minima}

Searching for minima\footnote{or equivalently maxima but due to the context of atomic simulations the discussion will be on minima.} is an important subject as they represent stable structures on PESes.
Using the available local information, the energy and the force, it is possible to follow the force with an appropriate step size until the gradient is zero, at which time a stationary point has been reached.
Generally this point will be a minimum but classifying non-minima will be discussed below.

Multiple techniques exist for performing these minimisations\footnote{Also often referred to as optimisations} in as few iterations as possible.
Some are as simple as adjusting the step size~\citemiss to simulate velocity, while others detect changes in this fictional velocity and adjust accordingly~\citemiss.
Still other methods build up information about the curvature to predict the appropriate step size~\citemiss and yet others make quadratic assumptions about the environment~\citemiss.

All these methods have a common theme of solving Newton's equations of motion with some type of fictional friction in order to reduce the kinetic and total energy of the system to bring it to a potential energy minimum.
