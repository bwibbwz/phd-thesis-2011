\section{Force}
\label{sec:force}
Since the PES, $E(\vR)$, is, generally, not known \textit{a priori}, traversing it is difficult without a guide.
Such a guide exists in the form of the gradient, $\nabla E(\vR)$, in which information about the increase/decrease of the function (here: PES) is encoded.
The gradient points towards the direction of greatest increase, however, in atomic simulations such areas are, commonly, of little interest.
Most of the time, systems are at, or near, stable structures, which are represented by minima on the PES.
Defining a force that, similarly to real world forces, points the system towards areas of lowest potential energy, is straight forwards as the negative gradient,
\beq{force-general}
\vF(\vR) = -\nabla E(\vR).
\eeq

The gradient is not directly available from DFT calculations, as for other non-analytical PESes.
Fortunately, for such variational calculations, a force theorem has been developed, stating that $\partial E / \partial\lambda$ can be evaluated for for a continuous parameter $\lambda$ on which the wavefunction, $\Psi(\lambda)$, depends implicitly,
\beq{hellmann-feynman-forces-full}
\frac{\partial E}{\partial \lambda} = 
\bra \frac{\partial \Psi}{\partial \lambda} | \widehat{H} | \Psi \ket +
\bra \Psi | \frac{\partial \widehat{H}}{\partial \lambda} | \Psi \ket +
\bra \Psi | \widehat{H} | \frac{\partial \Psi}{\partial \lambda} \ket.
\eeq
Utilising the orthogonality of the wave function, the first and last terms are zero, 
\beq{hellmann-feynman-forces-final}
\frac{\partial E}{\partial \lambda} = \bra \Psi | \frac{\partial \widehat{H}}{\partial \lambda} | \Psi \ket.
\eeq
Once the electronic wave function has been constructed, which is the most time consuming part, the gradient can be calculated.

For DFT, and any other Born-Oppenheimer calculations, $\lambda$ can be the nuclear coordinates. \tblue{Write out the force, medium priority.}
%\beq{dft-nuclear-forces}
%\vF_\lambda = \frac{\partial E}{\partial \vR_\lambda} = \bra \Psi | \frac{\partial \widehat{H}}{\partial \vR_\lambda} | \Psi \ket = 
%\eeq
Since the forces depend on the variational determination of the wave function (or electron density), their accuracy cannot be guaranteed and they tend to have numerical noise\citemiss which must be considered when using the forces.

The theorem has been proved independently by multiple authors~\cite{forces-pauli-1933, forces-guttinger-1932} but is generally attributed to Richard Feynman\cite{forces-feynman-1939} and Hans Hellmann~\cite{forces-hellmann-1937}, as the Hellmann-Feynman Theorem.
