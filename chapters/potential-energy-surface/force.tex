\section{Force}
\label{sec:force}

\bit
\item Introduce the force as the negative gradient
\item Relate the gradient to atomic force in reality, not just as the negative gradient.
\item Quickly discuss Hellmann-Feynman forces (if it can be done before the DFT section or if the DFT section is moved ahead of this one)
\eit

Since the PES, $E(\vR)$, is, generally, not known \textit{a priori}, traversing it is difficult without a guide.
Such a guide exists in the form of the gradient, $\nabla E(\vR)$, in which information about the increase/decrease of the function (here: PES) is encoded.
The gradient points towards the direction of greatest increase, however, in atomic simulations such areas are, commonly, of little interest.
Most of the time, systems are at, or near, stable structures, which are represented by minima on the PES \expand



