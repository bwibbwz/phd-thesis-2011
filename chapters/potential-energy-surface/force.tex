\section{Force}
\label{sec:force}
Since the PES, $E(\vR)$, is, generally, not known \textit{a priori}, traversing it is difficult without a guide.
Such a guide exists in the form of the gradient, $\nabla E(\vR)$, in which information about the increase/decrease of the function (here: PES) is encoded.
The gradient points towards the direction of greatest increase, however, in atomic simulations such areas are, commonly, of little interest.
Most of the time, systems are at, or near, stable structures, which are represented by minima on the PES.
Defining a force that, similarly to real world forces, points the system towards areas of lowest potential energy, is straight forwards as the negative gradient,
\beq{force-general}
\vF(\vR) = -\nabla E(\vR).
\eeq

The gradient is not directly available from DFT calculations, as for other non-analytical PESes.
Luckily, the fine work of Feymann and Hellmann allows \expand


