% ------------------------------------------------------------------
\section{The Born-Oppenheimer Approximation}
\label{sec:born-oppenheimer}

In order to simplify --- and in many cases, make possible --- quantum calculations of large atomic systems, the difference in weight of the electrons and the nuclei\footnote{\mytilde 4 orders of magnitude for hydrogen and more for the heavier elements.} is exploited by performing the calculations in two steps.
First, while the nuclei are kept motionless by excluding their kinetic operator from the Hamiltonian (\fref{eq:hamiltonian}), the electronic wavefunction and energy are determined followed by a calculation for the motion of the nuclei.
The assumption is, essentially, that for any motion of the nuclei, the electrons will move instantly, and remain in their ground state, to accommodate and is commonly known as the Born-Oppenheimer approximation\cite{born-oppenheimer-1927} (BOA).

\figmiss{Example PES}

This decoupling allows for a mapping of the potential energy as a function of the nuclear coordinates (commonly referred to as the potential energy surface or PES), as opposed to the continuum of PESes which exist should the motion of the nuclei and electrons be determined simultaneously.
PESes are, generally, not known \textit{a priori} and much effort is spent on traversing them to discover interesting features, such as minima, which represent stable atomic structures, or, as we shall see below, reaction pathways.

\tred{Stuck, moving on for now. This is usable now (apart from the lack of a figure) but would be nice to have a bit more. Medium importance}

\url{http://www.search.com/reference/Born-Oppenheimer_approximation}

%\bit
%\item For many systems this is a good approximation and has even proved useful even for metallic systems --- for which its use could be considered questionable due to the lack of a band gap ---
%\item For systems where there is a considerable gap between electronic states, larger than the energy scale of the nuclear motion, the BOA has been essential 
%\item Excited electronic states
%\item Good when the gap between electronic states is larger than the energyscale of the nuclear motion.
%\item Should be bad for metallic systems but has proved useful nevertheless. \footnote{see \url{http://www.nature.com/nmat/journal/v6/n3/pdf/nmat1846.pdf}}
%\item \url{http://www.jhu.edu/~chem/yarkony/research.html} specializes in non-BOA calculations.
%\eit

All the work carried out in this thesis employs the BOA.
