\section{Quantum Calculations}
\label{sec:methods-qm}

% Add a footnote about the spin density DFT (see Kieron's stuff for a reference)
% Also footnote on degeneracy

With increased computing power and, in particular, advances in parallel algortims, it is now possible to calculate a range of physical properties for materials on the atomic level without the use of any experimental parameters.
Such \textit{ab initio} methods are generally attempts at solving or approximating the Schr\"odinger equation (\fref{sec:schrodinger}) for the ground state\footnote{Non ground state calculations are not considered in this thesis.}.
Multiple strategies exist --- choosing one generally involves considering the trade-off between accuracy and computational effort --- but the main one used in this thesis is the Density Functional Theory, where the many-electron problem is \emph{exactly} divided into multiple one-electron problems.
\expand?

\subsection{The Electronic Structure Problem}
Solving the Schr\"odinger equation (\fref{sec:schrodinger}) is not a menial task, as it is a second order partial derivative equation in multiple dimensions and can only be solved exactly for very small systems.
Due to the high dimensionality of the wavefunction, simply storing it for a medium-to-large system is physically impossible.\footnote{Consider a system of only $16$ electrons (e.g. an oxygen molecule), sampled on a grid of $10$ points in each direction. Since each electron has 3 degrees of freedom the number of gridpoints will be $10^{16\times3} = 10^{48}$. Compared with the estimated amount of atoms in/on the Earth, $\mytilde10^{50}$\citemiss, the amount of gridpoints is impossible to store, even for a modest system.}

The Born-Oppenheimer approximation (\fref{sec:born-oppenheimer}) allows the separation of the Hamiltonian to an electronic part and a nuclear part.
Thus, the positions of the nuclei, $\vR_I$ only enter the calculations as parameters.
A full quantum treatment will not be discussed here (see for example~\citemiss).

A non-relativistic system of $N_e$ electrons (and $N_n$ nuclei) is described by the time independant Schr\"odinger equation~\cite{schrodinger-equation-1926}
\beq{schrodinger-basic-again}
 \widehat{H}\Psi = E\Psi,
\eeq
where $\Psi \equiv \Psi(\vr_1, \vr_2, \ldots ,\vr_{N_e})$ is the wave function, depending on the spatial coordinates, $\vr_i$, of each electron, $E$ is the electronic energy of the system and $\widehat{H}$ is the hamiltonian,
\beq{hamiltonian-full}
\widehat{H} = -\frac{1}{2} \sum_i^{N_e} \nabla^2_i + \sum_i^{N_e} \sum_{j>i}^{N_e} \frac{1}{\left| \vr_i - \vr_j \right|} + \sum_i^{N_e} \sum_I^{N_n} \frac{1}{\left| \vr_i - \vR_I \right|} + \widehat{V}_\text{env},
\eeq
which consists of 4 terms.
First there is the kinetic energy of the electrons, $\widehat{T}_\text{e} = -2^{-1}\sum_i^{N_e}\nabla_i^2$,
then the electron-electron interaction, $\widehat{U}_\text{ee} = \sum_i^{N_e}\sum_{j>i}^{N_e}(\left| \vr_i - \vr_j \right|)^{-1}$,
followed by the electron-nuclei interaction, $\sum_i^{N_e}\sum_I^{N_n}(\left| \vr_i - \vR_I \right|)^{-1}$,
and finally any further environmental influence (e.g. an applied electric field) is represented by a single term, $\widehat{V}_\text{env}$.
The last two terms are often bundled into a single external term,
\beq{hamiltonian-external}
V_\text{ext} = \sum_i^{N_e} \sum_I^{N_n} \frac{1}{\left| \vr_i - \vR_I \right|} + V_\text{env},
\eeq
since these are system dependant, while $\widehat{T}_\text{e}$ and $\widehat{U}_\text{ee}$ can be considered universal.

\subsection{Density Functional Theory}
\label{sec:methods-dft}
One of the most popular and common teqchniques for solving the electronic structure problem is the Density Functional Theory (DFT) which can exactly (in theory) solve the Schr\"odinger equation.
However, in practice, approximations must be made for all but the most simple systems.
Nevertheless, DFT has produced many interesting and good results and is now a well established formalism that is continually being reviewed and improved.
Furthermore, the \textit{ab initio} nature of DFT makes it a great companion to experimental research, not only for comparison and prediction of interesting materials but also to assist and expand upon data analysis.

The success of DFT yielded one of its founders, Walter Kohn, one half of the Nobel Prize in chemistry in 1998, for the development of the density functional theory~\cite{kohn-1999} along with John Pople for his development of computational methods in quantum chemistry~\cite{pople-1999}.

\subsubsection{The Hohenberg-Kohn Theorems}
The earliest mention of a DFT type method were made by Thomas~\cite{thomas-1927} and Fermi~\cite{1927} but the heart of DFT lies in the Hohnberg-Kohn theorem, which reformulates the electronic structure problem to depend on the $3$ dimensional, ground state, electron density, $\rho_0(\vr)$, instead of the $3N$ dimensional wavefunction, $\Psi(\vr_1, \vr_2, \ldots, \vr_N)$.
According to the first Hohenberg-Kohn theorem~\cite{hohenberg-kohn-1964}, any external term, $V_\text{ext}$, in \fref{eq:hamiltonian-full} yields a different electron density from any non-identical $V'_\text{ext}$ up to trivial additive constant, for the ground state.
In other words: there is a one-to-one correspondence between the wavefunction and the electron density of the ground state and all observables that depend on the wavefunction can, thus, be extracted from the ground state density.
This allows a re-formulation of the Schr\"odinger equation, with the energy as a functional of the electron density, 
\beq{reformulated-schrodinger-equation}
E[\rho_0] = \bra \Psi[\rho_0] | \widehat{H} | \Psi[\rho_0] \ket.
\eeq

Furthermore, the second Hohenberg-Kohn theorem~\cite{hohenberg-kohn-1964} adds a variational~\cite{variational-rayleigh-1870, variational-ritz-1909} way to discover the true ground state density and ground state energy,
\beq{variational-density}
E_0 = \min_\rho E[\rho].
\eeq

This work alone is still missing key elements in order to efficiently solve for the wavefunction and energy, namely the wavefunction to density mapping is not known.

\subsubsection{The Kohn-Sham Equations}
Counter-intuatively re-introducing wave functions was the solution proposed by Walter Kohn and Lui Jeu Sham in 1965~\cite{kohn-sham-1965}.
Inspired by Hartree's work nearly four decades earlier~\cite{hartree-1928}, multiple non-interacting single particle wave functions, $\phi_i(\vr_i)$ took the place of the many-body wave function (or electron density), such that
\beq{kohn-sham-density}
 \rho(\vr) = \sum_{i=1}^{N_e} \left| \phi_i(\vr_i) \right|^2,
\eeq
where each electron would be subject to an effective potential, $v(\vr)$, due to all the particles in the system (including itself, the nuclei and any external factors).
Solving for $\phi_i$ and $\epsilon_i$ (the Kohn-Sham energies) using multiple Schr\"odinger type equations (commonly referred to as the Kohn-Sham equations),
\beq{kohn-sham-equations}
 \left\{-\frac{1}{2} \nabla ^2 + v(\vr) \right\} \phi_i(\vr_i) = \epsilon_i \phi_i(\vr_i),
\eeq
can be done iteratively in a variational manner, from an initial guess for the electron density.

Constructing the effective potential becomes the main problem here, since an exact $v(\vr)$ yields an exact solution to the whole electronic structure problem.
The external potential, $\widehat{V}_\text{ext}$, from \fref{eq:hamiltonian-external} enters unchanged as it contains only static information.
Conversely, the electron-electron interaction, $\widehat{U}_\text{ee}$, enters in a changed form.
In fact, it is introduced as two terms, one for interaction with the electron density as a whole, commonly referred to as the hartree potential ($v_\text{H}(\vr)$) and a second one for many-electron and Pauli-exclusion effects, %self-interaction effects, 
$v_\text{XC}(\vr)$, the so-called exchange-correlation potential,
\beq{effective-potential}
v(\vr) = V_\text{ext}(\vr) + v_\text{H}(\vr) + v_\text{XC}(\vr).
\eeq
The hartree potential is an integral over the electron density,
\beq{hartree-potential}
v_\text{H}(\vr) = \int \frac{\rho(\vr')}{\left| \vr - \vr' \right|}d^3\vr',
\eeq
and the exchange-correlation potential is of the form,
\beq{exchange-correlation-potential}
v_\text{XC}(\vr) = \frac{\partial E_{XC}[\rho(\vr)]}{\partial \rho(\vr)}.
\eeq

Thus far the formalism is exact, however, all the difficult parts, $E_\text{XC}[\rho(\vr)]$, have yet to be discussed.

\subsubsection{The Exchange-Correlation Functional}
The Kohn-Sham equations are in principle exact, however, the exchange-correlation functional, $E_\text{XC}[\rho(\vr)]$, is generally not known and has to be approximated.
The accuracy of the this approximation is a major factor in determining the accuracy of DFT as a whole.
A detailed discussion of Exchange-Correlation functionals is material enough for a whole chapter, thus only the basic approximations will be discussed here.

The Local Density Approximation (LDA)~\cite{kohn-sham-1965} treats exchange-correlation as if it were a homogeneous electron gas,
\beq{xc-lda}
E_\text{XC}^\text{LDA}[\rho(\vr)] = \int \epsilon_\text{XC}^\text{hom.}[\rho(\vr)]\rho(\vr)d^3\vr,
\eeq
where $\epsilon_\text{XC}\text{hom.}$ \expand

%The approximations used in our work are based on the so-called Generalized Gradient Approach (GGA). The GGA uses the the exchange-correlation energy of a homogeneous electron gas at point $\vect{r}$ like the Local Density Approximation (LDA) \cite{kohn1965} but also uses the gradient of the density to account for inhomogeneity. This, however, is not a good choice by itself so a reduced density gradient, $s(\vect{r})$, is used, as suggested by Langreth and Perdew \cite{langreth1977}. The exchange-correlation functional then becomes
%\begin{equation}
%  E_{XC}^{GGA}[\rho] = \int d^3r\, \epsilon_{XC}^{GGA}(\rho(\vect{r}), s(\vect{r}))\rho(\vect{r})
%\end{equation}
%with
%\begin{equation}
%\label{eq:GGAreducedDensityGradient}
% s(\vect{r}) = \frac{\left|\nabla \rho(\vect{r})\right|}{2\sqrt[3]{3\pi^2\rho(\vect{r})} \rho(\vect{r})}.
%\end{equation}

%\subsection{Implementation of DFT}
%\label{sec:methods-dft-implemetnation}
%
%\subsubsection{The Plane Wave Basis Set}
%In implementing DFT calculations, the Kohn-Sham wavefunctions are expanded in a particular basis set. In our work plane waves are used under periodic boundary conditions in accordance with Bloch's theorem,
%\begin{equation}
%\label{eq:blochsTherom}
% \psi_\vect{k}^{m}(\vect{r}) = \sum_\vect{G}c_{\vect{k} + \vect{G}}^m\, e^{i(\vect{k}+\vect{G})\cdot\vect{r}}
%\end{equation}
%where $\vect{G}$ are the reciprocal lattice vectors. %[what are $c_{\vect{k} + \vect{G}}^m$?]. 
%For an exact solution, an infinite number of plane waves is needed. Fortunately the plane waves at the lower end of the kinetic energy range are the most important, so the number of plane waves can be reduced by defining a cutoff, $G_{cut}$, where the solution becomes good enough:
%\begin{equation}
% \left( \frac{\hbar^2}{2m} \right) \left| \vect{k} + \vect{G}_{cut} \right|^2 \le E_\mathrm{cut}.
%\end{equation}
%This leads to one of the main advantages of plane waves. By increasing the cutoff the accuracy of the calculation can be systematically increased. 
%
%A disadvantage of plane waves is their inefficiency to deal with high curvature regions, such as the atomic core. To overcome this pseudopotentials can be used. In the core, pseudopotentials are an average of the potential due to the core electrons and the nucleus felt by the valence electrons inside a given sphere but outside the sphere the pseudopotential becomes identical to the all-electron potential. The plane-wave cutoff can be lowered even further when using pseudopotentials while generally giving results of good accuracy, especially the so-called ultrasoft pseudopotential with non-local components \cite{vanderbilt1990}.
%
%\subsection{Known Problems of DFT}
%\bit
%\item Band Gaps in semi conductors
%\item Dispersion (van Der Waals)
%\item etc.
%\eit
%
%\placeholder
