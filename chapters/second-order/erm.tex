\section{Ridge Mapping}
The goal is to define a path which can be iteratively optimized to align itself with a specific steepest decent path (SDP),
starting at a \sap2 and ending at a \sap1.
To accomplish this a path between neighboring \sap1s is defined,
which once converged will lead from one \sap1 along an SDP,
through a \sap2,
continuing along an SDP to, finally, arrive at the latter \sap1.
The path represents two SDPs, from the \sap2 to each \sap1 respectively.

For a function $V$ of multiple variables, $\vR$, a ridge is a collection of points where the gradient, $\vF \equiv -\nabla V(\vR)$, prependicular to the path, defined by its tangent, $\uvt$,
\beq
\vF^\perp \equiv \vF - (\vF \cdot \uvt)\uvt,
\eeq
vanishes,
\beq
\vF^\perp_\text{@ridge} = \vect{0}.
\eeq
To distinguish ridges from other SDPs,
a condition for the Hessian must be set,
exactly one negative eigenvalue, $\lambda^\text{min}$, must be present in a reduced Hessian, for the subspace excluding the path's tangent,
corresponding to an eigenvector, $\uvm$, that is perpendicular to the path,
\beq
\uvm \cdot \uvt = 0 \quad \text{and} \quad \lambda^\text{min} < 0.
\eeq




\incomplete

\subsection{Energy Ridge Mapping}
In the context of atomic simulations, the function in question is often the potential energy as a function of all atomic co-ordinates.

\incomplete

