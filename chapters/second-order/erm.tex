\section{Ridge Mapping}
The goal is to define a path which can be iteratively optimized to align itself with a specific steepest decent path (SDP),
starting at a \sap2 and ending at a \sap1.
To accomplish this a path between neighboring \sap1s is defined,
which once converged will lead from one \sap1 along an SDP,
through a \sap2,
continuing along an SDP to, finally, arrive at the latter \sap1.
The path represents two SDPs, from the \sap2 to each \sap1 respectively.
%We shall refer to this pair of SDPs as a ridge.

For a function $V$ of multiple variables, $\vR$, a ridge is a collection of points where the gradient, $\vF \equiv -\nabla V(\vR)$, prependicular to the path,
\beq
\vF^\perp \equiv \vF - (\vF \cdot \uvt)\uvt,
\eeq
vanishes,
\beq
\vF^\perp_\text{@ridge} = \vect{0},
\eeq
where $\uvt$ is the tangent of the path,
and for a reduced Hessian matrix, where any components along the path have been removed, only one eigenvalue is negative, corresponding to an eigenvector that is perpendicular to the path.




\incomplete




\subsection{Energy Ridge Mapping}
In the context of atomic simulations, the function in question is often the potential energy as a function of all atomic co-ordinates.





\incomplete



