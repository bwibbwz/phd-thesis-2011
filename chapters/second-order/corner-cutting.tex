\section{Corner Cutting}
\label{sec:erm-corner-cutting}

Should finding the exact ridge be the goal --- rather than just the \sap{2} and a ridge estimate --- and the alternative tangent discussed above not be sufficient, multiple strategies can be envisioned, all of which aim at somehow reducing the perpendicular component of the spring force while avoiding or minimising the instabilities discussed above.

\subsubsection{What is corner cutting}
The spring force tries to keep each image as close as possible to its neighbours.
Without any further influence, the images would align evenly spaced in a straight line.
However, since there are other forces at work, perpendicular to the path, $\vF^\text{t}$, an equilibrium between the conflicting forces will be reached, when they are equal in length but with opposite direction,
\beq{corner-cutting-equilibrium}
\vF^{\text{S}\perp} = -\vF^\text{t},
\eeq
producing a systematic error which is commonly referred to as corner cutting.
Since the goal of the whole method is to fulfil \fref{eq:ridge-force}, such an equilibrium can be unsatisfactory.

This problem existed in the infancy of NEB calculations but was made a thing of the past with more numerically suitable tangent and spring force definitions.~\cite{neb-tangent-2000}
These changes are not directly applicable to ridge calculations due to intrinsic stability issues which are dampened by the perpendicular component of the spring force.
However, once the path is sufficiently well behaving and near the ridge, minimising the corner cutting could be done.

\figmiss{Graphic representation of corner cutting. Medium priority.}

\subsection{Minimise corner cutting}
\textit{After publishing paper \ref{pap:second-order} some effort went into finding the exact ridge instead of paths that suffer from corner cutting.
The methods and ideas presented here serve more as a detailed outlook for future work than a complete study.}
\vspace{1em}

In order to reduce/remove corner cutting, the perpendicular component of the spring force must be removed/reduced at each image.
Then in tune with the equilibrium presented in \fref{eq:corner-cutting-equilibrium}, \fref{eq:ridge-force} will be fulfilled and the exact ridge found.

Using the dual tangent scheme, presented above, it is possible to significantly reduce the numerical instabilities of the ridge calculation.
In some test cases even allowing for the complete exclusion of the perpendicular spring force after turning on the climbing image.
However, should this not be possible, a multiplication factor, $\xi_i \in [0, 1]$, for the perpendicular spring force,
\beq{spring-force-perpendicular}
\vF_i^{\text{S}\perp} = \vF_i^\text{S} - (\vF_i^\text{S} \cdot \uvt_i)\uvt_i,
\eeq
can be defined,
\beq{spring-force-reduction}
\vF_i^{\text{S, eff.}\perp} = \xi_i \vF_i^{\text{S}, \perp},
\eeq
to iteratively bring the effective perpendicular component, $\vF_i^{\text{S, eff.}\perp}$, to zero and thus converging to the exact ridge.

Multiple schemes for reducing $\xi_i$ can be envisioned.
One, that initial testing showed to be successful, is reducing $\xi_i$ each time corner cutting is detected and then continuing the iterative convergence until either corner cutting is detected again or the path is fully converged to the ridge.
By how muchi $\xi_i$ is reduced each time, could either be a fixed ratio or, preferably, dependant on the amount of corner cutting.\footnote{Only the fixed ratio scheme was tested.}
In order for the path to converge evenly, $\xi_i$ must be constricted to be within a certain range of $\xi_{i-1}$ and $\xi_{i+1}$.
Using a scheme such as this, the corner cutting was significantly reduced in the test systems\footnote{Both custom potentials and the Al self diffusion (\fref{chap:al}) systems were tested but not extensively.} it was applied to.

It warrants repeating that the ridge method, as presented originally in paper \ref{pap:second-order}, is able to find the \sap{2} exactly without problem but if the ridge is curved, corner cutting will take place.
While the removal of corner cutting would present a truer picture of the ridge, the method as it stood is still of good use.

%The main problem with simply removing the perpendicular spring force is the instabilities that exist due to the orthogonality constraint (see above) of \fref{eq:orthogonality-constraint}, which in turn is employed to overcome other instabilities.

\begin{comment}
\subsection{Older stuff}

\bit
\item Iteratively remove parts of $\vF^\text{S}_\perp$ and relax
\item Detect corner cutting ($\hat{\vF^\text{t}_\perp} \cdot \hat{\vF^\text{S}}_\perp \approx -1$ and $|\vF^\text{t}_\perp| \approx |\vF^\text{S}_\perp|$). Maybe detect a cone around the corner cutting direction and reduce containment there for the next steps also.
\eit
I've tried both with moderate to good success rate on the hop diffusion of Al (EMT).
Need to move on to more complicated systems (concerted self-diffusion of Al).

--- Minor fluctuations in the tangent ... 

A possible "solution" is to keep the dimer fixed after switching to the "reduce corner-cutting phase" but then the dimer WILL be wrong after a few iteration in the last phase.
Maybe the answer is to take an average of previous tangents, etc.

Since there is no guarantee that the eigenmodes align with the tangent and minimum mode at all points on the ridge (see \fref{fig:modes}), it is not advisable to let go of the orthogonality constraint between the minimum mode and the tangent,
\beq{orthogonality-constraint}
\uvn \cdot \uvt = 0.
\eeq
Early tests of releasing the constraint on some of the images were not reliable.

\incomplete

\subsubsection{\tblue{Informal Babble}}

I have two ideas that I'm currently working on.
\paragraph{The first scheme}
By gradually removing $\vF^{\text{S}\perp}$, the path is brought closer to the true path.
There are multiple ways to do this but the way I do it currently is to detect the corner cutting and its direction.
When the perpendicular spring force and the perpendicular transformed force are pointing in opposite directions,
\beq{asdf1}
\frac{\vF^{\text{t}\perp}_i}{|\vF^{\text{t}\perp}_i|} \approx -\frac{\vF^{\text{S}\perp}_i}{|\vF^{\text{S}\perp}_i|},
\eeq
and their length is the same,
\beq{asdf2}
|\vF^{\text{t}\perp}_i| \approx |\vF^{\text{S}\perp}_i|,
\eeq
corner cutting has taken place.
When these criteria are satisfied, $\vF^{\text{S}\perp}_i$ is permanently decreased by a factor, $\xi_i$,
\beq{asdf3}
\vF^{\text{S}\perp\text{eff}}_i = \vF^{\text{S}\perp\text{full}}_i \xi_i.
\eeq
Currently there is no good way to decide how much $\xi_i$ is decreased when corner cutting is detected but it should be easy to devise a scheme.

This has worked well on the hop-hop test system in multiple dimensions to reduce the corner cutting significantly (90\%) but not fully as the band becomes unstable (in particular when many images are used, above 40).

\paragraph{The second scheme}
It seems that there are two conflicting factors that hinder finding the true path.
\bit
\item When the tangent and dimer are kept perpendicular to each other, kinks seem to form in the band (I have not managed to fully understand why) and once this happens, everything goes bad quickly
\item This  seems to be less of a problem when the dimer is allowed to freely rotate but allowing this in systems where the true eigenmodes do not align with the ridge yields forces that quickly pull the band far off the ridge. 
\eit

Initial testing (on the hop-hop system) shows that changing the eigenmode-tangent orthogonality constraint to use a different tangent ($\vR_{i+1} - \vR_{i-1}$) which is more similar to the real ridge when an image gets slightly off the ridge, yields up to $100\%$ reduction of corner cutting, when used along with a complete removal of $\vF^{\text{S}\perp}$.

The current tangent (which only depends on the displacement to the top energy image) is less accurate but works better due to stability issues (kinks do not form).

It does seem that alining the dimer perpendicular to the ridge is essential because otherwise the significantly sized force that lies along the ridge (and should get "nudged" out) get deflected and produces a force in the wrong direction.

\paragraph{Scheme 1.5}
How about applying the first schem from the Saddle point downwards.
Either converging points $i_\text{max} \pm 1$ and then locking them in place (similar to the CI) and then converging points $i_\text{max} \pm 2$, etc.

OR using the first cheme but never allowing constricting $\xi_i$ to be higher than for the neighboring higher energy image.

It is very easy to try this latter implementation but the first one requires a bit of programming.

\paragraph{Measuring corner cutting}
The corner cutting is measured by dotting together the effective force and the tangent,
\beq{asdf4}
\text{measure} = \frac{\vF_i^\text{eff}}{|\vF_i^\text{eff}|} \cdot \uvt,
\eeq
When this is $1$, there is no corner cutting.

\end{comment}
