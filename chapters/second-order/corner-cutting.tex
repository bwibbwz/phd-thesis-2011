\section{Minimising Corner Cutting}
\label{sec:erm-corner-cutting}
% ------------------------------------------------------------------

Should finding the exact ridge be the goal, multiple strategies can be envisioned all of which aim at somehow reducing the perpendicular component of the spring force while avoiding and/or minimising the instabilities discussed above.

\bit
\item Iteratively remove parts of $\vF^\text{S}_\perp$ and relax
\item Detect corner cutting ($\hat{\vF^\text{t}_\perp} \cdot \hat{\vF^\text{S}_\perp \approx -1}$ and $|\vF^\text{t}_\perp| \approx |\vF^\text{S}_\perp|$). Maybe detect a cone around the corner cutting direction and reduce containment there for the next steps also.
\eit

--- Minor fluctuations in the tangent ... 

A possible "solution" is to keep the dimer fixed after switching to the "reduce corner-cutting phase" but then the dimer WILL be wrong after a few iteration in the last phase.
Maybe the answer is to take an average of previous tangents, etc.

Since there is no guarantee that the eigenmodes align with the tangent and minimum mode at all points on the ridge (see \fref{fig:modes}), it is not adviable to let go of the orthogonality constraint bwteen the minumum mode and the tangent,
\beq{orthogonality-constraint}
\uvn \cdot \uvt = 0.
\eeq

\incomplete
