\section{Introduction}
\label{sec:erm-introduction}
% -----------------------------------------------------------------

A collection of two specific steepest descent paths starting infinitesimally close to a \sap{2} and ending at two neighbouring \sap{1}s can be considered a ridge.
Such a path is non-trivial to find as information about the curvature, the Hessian, is essential but often unavailable in a direct manner.

A given point, $\vR$, is located on a ridge if the gradient is parallel with its tangent and the Hessian matrix for the vector space perpendicular to the tangent has a single negative eigenvalue~\footnote{Higher order "ridges" can be identified if the Hessian has more negative eigenvalues.}.

A method for iteratively aligning a path with the ridge is presented, building on the well established NEB (\fref{sec:neb}) and Dimer (\fref{sec:dimer}) methods.
This method is then applied to the self diffusion of an adatom on the \ce{Al}(100) surface in \fref{chap:al}.

After converging to the ridge, the validity of the harmonic approximation to transition state theory (HTST) is considered and an improvement to the reaction rate offered.

