\section{Ridge Mapping}
\label{sec:ridge-mapping}

Given a function, $E$, of multiple variables, $\vR$, its gradient, $\nabla E(\vR)$, and two \sap{1}s, the goal is to identify a path that lies close to the ridge between the two \sap{1}s.
The path should, in particular, lie through any intermediate \sap{2}s so that a comparison of the height of \sap{2}s with respect to the \sap{1}s can be made.
The method should, furthermore, lead to the identification of previously unknown \sap{1}(s) on the ridge in between the given end points, should they exist.

\subsubsection{Gradient Modification}
Similarly to the NEB method~\cite{neb-original-1998} (\fref{sec:neb}), a path of $N$ discrete images, $[\vR_0, \vR_1, \ldots, \vR_N]$, is iteratively aligned with the ridge by modifying the gradient or force, $\vF_i \equiv -\nabla E(\vR_i)$, for each one.
However, unlike the NEB method, further force modifications are needed, in order to let the ridge appear as a MEP.

The path is at each image defined by its tangent, $\uvt_i$, and in order for the path to be at a SDP (such as the ridge), any force components perpendicular to it,
\beq{perpendicular-force}
\vF_i^\perp \equiv \vF_i -(\vF_i \cdot \uvt_i)\uvt_i,
\eeq
must be zero,
\beq{ridge-force}
\vF^\perp_\text{ridge} = \vect{0}.
\eeq
Furthermore, one negative eigenvalue of the Hessian, perpendicular to the path, must be guaranteed.\footnote{If finding higher order ridges is desired, more negative eigenvalues must be guaranteed.}
The Dimer methodology~\cite{dimer-original-1999, dimer-olsen-2004} (\fref{sec:dimer}) for finding the lowest eigenvalue and the corresponding eigenvector (minimum mode) of the Hessian, $\uvn_i$, can produce a transformed force,
\beq{transformed-force}
\vF_i^\text{t} = \vF_i^\perp - 2(\vF_i^\perp \cdot \uvn_i)\uvn_i,
\eeq
that will let ridges appear, locally as MEPs.
Since eigenvectors are not necessarily perpendicular to the ridge (as can be seen in figure 1 of paper \ref{pap:second-order}), the path itself must be left out of the Hessian's vector space, thus applying an orthogonality constraint on the minimum mode,
\beq{orthogonality-constraint}
\uvt_i \cdot \uvn_i = 0,
\eeq
at all times.

Similarly to the NEB method, an artificial force is employed in order to keep the images equally\footnote{or with controlled spacing} distributed along the path.
For the NEB method this force acts only along the path and is generally referred to as the spring force as it resembles springs connecting the images.
However in the case of ridge calculations, the full spring force most be used,
\beq{full-spring-force}
\vF_i^\text{S} = k \left[ \left( \vR_{i+1} - \vR_i \right) - \left( \vR_i - \vR_{i-1} \right) \right],
\eeq
where $k$ is the spring constant, which controls the stiffness of the springs.
Retention of the full spring force is necessary due to numerical instabilities which proved to be more prominent in ridge calculations than MEP calculations.

Combining the above forces, $\vF_i^\text{t}$ and $\vF_i^\text{S}$, into an effective force (as shown in \fref{fig:erm-force-diagram}),
\beq{erm-effective-force}
\vF_i^\text{eff} = \vF_i^\text{t} + \vF_i^\text{S},
\eeq
will allow the path iteratively converge to a close to the ridge.

\subsubsection{Numerical Instabilities}
%\bit
%\item The orthogonality constraint, introduced in \fref{eq:orthogonality-constraint}, has the possibility to introduce numerical instabilities.
%\item Simply using the dimer introduces instabilities.
%\item Odd starting profiles (monotonic or inverted barriers) are dangerous.
%\eit

Simply using the Dimer, or any other non-exact, estimate of the minimum mode introduces instabilities in the forces, as the ridge$\rightarrow$MEP mapping can get inaccurate.
This, directly, is not a problem hindering convergence since the Dimer generally has ample time to converge to the nearly exact minimum mode.
More serious, seems to be the interaction between the minimum mode and the tangent, as enforced by \fref{eq:orthogonality-constraint}.
The tangent is designed to minimise numerical instabilities in the NEB~\cite{neb-tangent-2000} but not to offer a particularly good estimate of the path's tangent at any given time, however, once converged, it offers a reasonably good estimate\footnote{A better estimate of the steepest descent path in question (the minimum energy path) would be a tangent pointing downwards in energy rather than upwards.} and in the limit of infinite amount of images, an exact tangent.
Using a tangent that depends on both neighbouring images, the path would form kinks under minute perturbations that would not even out with more iterations.
The orthogonality constraint enforces an equally wrong minimum mode estimate under perturbations of the path.
This, in turn, introduces similar instabilities as the kinks, when the ridge$\rightarrow$MEP mapping becomes increasingly inaccurate (\fref{fig:dimer-instabilities}).
In paper \ref{pap:second-order}, these instabilities are countered by the inclusion of the perpendicular spring force component which yield a more systematic error of corner cutting (discussed below).
Further analysis of these instabilities showed that enforcing the orthogonality constraint (\fref{eq:orthogonality-constraint}) using a tangent estimate that is focused on accuracy, rather than stability, highly reduces the problematic behaviour.
The system would then have two tangent definitions co-existing, a numerically stable one for the NEB-type force scheme (\fref{eq:perpendicular-force}) and a more accurate representation of the ridge,
\beq{central-difference-tangent}
\uvt_i^\text{ridge} = \frac{\vR_{i+1} - \vR_{i-1}}{\left| \vR_{i+1} - \vR_{i-1} \right|},
\eeq
for the space reduction of the Hessian (\fref{eq:orthogonality-constraint}).

This latter scheme was neither employed in paper \ref{pap:second-order} nor \fref{chap:al} and has not been tested fully.
Nevertheless, the initial results are promising for the removal/reduction of corner cutting.

Since the end points of the ridge are not local minima, there is no intrinsic energy barrier separating them, as is the case for NEB calculations.
In fact, the initial energy profile can have any shape, including a monotonic one or even an inverted barrier, if it lies near a local minimum, which is not unlikely.
This means that the initial environment of the path is drastically different from that of the ridge.
To help bring the path nearer to the ridge, the full spring force \expand





\subsubsection{Exact Convergence to the \sap{2}}
Beyond the problems already discussed regarding converging exactly to the \sap{1} in NEB calculations, systematic errors due to the retention of the full spring force makes it unlikely that using only \fref{eq:erm-effective-force} will yield an image at the exact \sap{2}.
As with the NEB method a Dimer-type solution is possible --- similar to equations \ref{eq:dimer-transform} and \ref{eq:neb-ci-transform} --- where the highest value image is decoupled from the springs and force components along the \emph{two} lowest eigenvalued eigenmodes, as defined by the minimum mode and the tangent,
\beq{erm-ci}
\vF_{i_\text{max.}}^\text{eff.} = \vF_{i_\text{max.}} - 2(\vF_{i_\text{max.}} \cdot \uvt_{i_\text{max.}})\uvt_{i_\text{max.}} - 2(\vF_{i_\text{max.}} \cdot \uvn_{i_\text{max.}})\uvn_{i_\text{max}},
\eeq
where $i_\text{max.}$ refers to the image with the highest functional value.

Unlike the NEB, the computational effort involved is not easily measured due to the systematic errors introduced by the full spring force but there is no reason to suspect increased effort by applying \fref{eq:erm-ci}.
Also, this addition can not be considered optional, as it can technically for NEB, as it serves as an indirect stabilisation tool at the top of the path, where the environment is concave\footnote{Curves downwards} in 2 dimensions rather than 1 as is the case with a MEP.

\subsection{Energy Ridge Mapping}
\label{sec:energy-ridge-mapping}
In the context of atomic simulations, the function in question is often the potential energy as a function of all atomic co-ordinates, or the Potential Energy Surface.

Information about energy ridges can be useful when considering reaction rates, e.g. for checking the validity of the harmonic approximation to TST (\fref{sec:htst}).
Furthermore, should HTST not be sufficient it is possible to make adjustments to the rate using information about the ridge.
This move beyond harmonicity is discussed below in \fref{sec:beyond-harmonicity}, in \fref{chap:al} and paper \ref{pap:second-order}.
