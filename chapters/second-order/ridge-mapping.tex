\section{Ridge Mapping}
\label{sec:ridge-mapping}

Given a function, $E$, of multiple varibles, $\vR$, its gradient, $\nabla E(\vR)$, and two \sap{1}s, the goal is to identify a path that lies close to the ridge on which the two \sap{1}s lie.
The path should, in particular, lie through any intermediate \sap{2}s so that a comparison of the height of \sap{2}s with respect to the \sap{1}s can be made.
The method should, furthermore, lead to the identification of previously unknown \sap{1}(s) on the ridge in between the given end points, should they exist.

\subsubsection{Gradient Modification}
Similarly to the NEB method~\cite{neb-original-1998} (\fref{sec:neb}), a path is iteratively aligned with the ridge by modifying the gradient or force, $\vF \equiv -\nabla E(\vR)$.
Unlike the NEB method, further force modifications are needed, in order to let the ridge seem as a MEP.

The path is defined by its tangent, $\uvt$, and in order for the path to be at a SDP (such as the ridge), any force components perpendicular to it,
\beq{perpendicular-force}
\vF^\perp \equiv \vF -(\F \cdot \uvt)\uvt,
\eeq
must be zero,
\beq{ridge-force}
\vF^\perp_\text{ridge} = \vect{0}.
\eeq
Furthermore, one negative eigenvalue of the Hessian, perpendicular to the path, must be guaranteed.\footnote{If finding higher order ridges is desired, more negative eigenvalues must be guaranteed.}
Utilising the Dimer methodology~\cite{dimer-original-1999, dimer-olsen-2004} (\fref{sec:dimer}) for finding lowest eigenvalue and the corresponding eigenvector (minimum mode), $\uvn$, a transformed force,
\beq{transformed-force}
\vF^\text{t} = \vF^\perp - 2(\vF^\perp \cdot \uvn)\uvn,
\eeq
will let ridges appear, locally as MEPs.
Since eigenvectors are not neccesarily perpendicular to the ridge (as can be seen in figure 1 of paper \ref{pap:second-order}), the path itself must be left out of the Hessian's vector space, thus applying an orthogonality constraint on the minimum mode,
\beq{orthogonality-contrainst}
\uvt \cdot \uvn = 0,
\eeq
at all times.

\subsubsection{Numerical Instabilities}
\bit
\item The orthgonality constraint, introduced in \fref{orthogonality-constraint}, has the possibility to introduce numerical instabilities.
\item Simply using the dimer introduces instabilities.
\item Odd starting profiles (monotonic or inverted barriers) are dangerous.
\eit

\placeholder

\subsection{Energy Ridge Mapping}
\label{sec:energy-ridge-mapping}

In the context of atomic simulations, the function in question is often the potential energy as a function of all atomic co-ordinates.

\incomplete
