\section{Energy Ridge Mapping}
\label{sec:energy-ridge-mapping}
%In the context of atomic simulations, the function under investigation is generally the PES.
%in question is often the potential energy as a function of all atomic coordinates, or the potential energy surface.
In the context of atomic simulations, information about energy ridges on the PES can be useful when considering reaction rates, e.g. for checking the validity of the harmonic approximation to TST (\fref{sec:htst}).
Furthermore, should HTST not be sufficient it is possible to make adjustments to the rate using information about the ridge.
This move beyond harmonicity is discussed below in \fref{sec:beyond-harmonicity}, in \fref{chap:al} and paper \ref{pap:second-order}.

\subsection{Beyond Harmonicity}
\label{sec:beyond-harmonicity}
The reaction rate offered by the, commonly used, harmonic approximation to transition state theory (\fref{sec:htst}), HTST, assumes two things.
First, that the transition state, $\ts$, can be described with a hyperplane in which the relevant \sap{1} lies and whose normal is the eigenvector corresponding to the negative eigenvalue of the Hessian at the \sap{1}.
Second, that the energy profile at the $\ts$ and minimum can be approximated with a second degree Taylor expansion.
These assumptions make the harmonic approximation faster than full TST by orders of magnitude.

The conditions under which HTST is valid are generally that the energy, $E(\vR)$, at the minimum of the reactants basin must be sufficiently lower than at the \sap{1} and that the \sap{1} energy must be sufficiently lower than the energy at any neighbouring \sap{2}s.
The literature is unspecific as to what a sufficient difference is but a commonly used value is that $E(\vR_{\sap{1}}) - E(\vR_\text{min.}) > 5\kB T$~\cite{htst-5ev-2005}.
It is reasonable to assume a similar criterion for $E(\vR_{\sap{2}}) - E(\vR_{\sap{1}})$.

Ridge calculations can be used to check if this criterion is fulfilled but further, they can be used to improve the reaction rate.
By comparing the configurational integral of the harmonic energy profile for a given degree of freedom, 
\beq{harmonic-configurational-integral}
Z_\ts^\text{harm.} = \int_{-\infty}^\infty e^{-\alpha x^2/ \kB T}dx,
\eeq
to that of the ridge,
\beq{ridge-configurational-integral}
Z_\ts^\text{ridge} = \int_\text{ridge} e^{-E(x)/ \kB T}dx,
\eeq
where x is the displacement along the particular degree of freedom under investigation and $\alpha$ is the corresponding Taylor coefficient, corrections can be made to the reaction rate.
The ratio of the configrational integrals,
\beq{htst-correction-factor}
\Gamma = \frac{Z_\ts^\text{ridge}}{Z_\ts^\text{harm.}},
\eeq
can be used as a multiplicative correction factor to the harmonic reaction rate,
\beq{corrected-htst-rate}
k_\text{HTST}^\text{corrected} = \Gamma k_\text{HTST},
\eeq
for each investigated degree of freedom.

Since the harmonic integral has a larger range, $[-\infty, \infty]$, than the ridge integral, $[\sap{2}^\text{A}, \sap{2}^\text{B}]$, it is possible for the ratio to become larger than 1.0.
Such ratios are artefacts and should not be used to increase the reaction rate.
It is possible to limit the harmonic integral to the same range as the ridge integral, using the error function, to avoid this.
In the tests presented in \fref{chap:al}, $\Gamma > 1.0$ was not a common nor large problem compared with the factors where $\Gamma < 1.0$ and thus the infinite limits were used.

Choosing the direction of $x$ is non-trivial.
In the work presented in \fref{chap:al}, it is implicitly chosen when $\alpha$ is prepared by means of a least squares analysis of the 4 lowest energy images of the ridge.
Another method would be to use the tangent of the ridge at the \sap{1} in question to extract the corresponding vibrational frequency from the Hessian but this method was neither tried nor tested.
In general, the accuracy of $\Gamma$ is not essential as it can only offer a rough correction estimate rather than a rigorous correction to HTST.
