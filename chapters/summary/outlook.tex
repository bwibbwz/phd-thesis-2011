\section{Outlook}
\label{sec:summary-outlook}

A detailed outlook as to further development of the ridge detection algorithm was presented in \fref{chap:erm}.


Finding further use for the energy ridges will be interesting.
For example, stitching together a transition state from hyperplanar segents near each image of every ridge (with the minimum modes as normals), where the amount of images could possibly viewed as a variational parameter in lieu of the transition state.


\bit
\item DFT ridges
\item How ridge calculations interact with different types of saddle points (such as those with $H = \vect{0}$, SPs which are also points of inflection or turning points)
\item Rigorous performance study.
\item Discover the limit of how low \sap{2} can be with regards to \sap{1}. i.e. confirm the $5\kB T$ stuff.
\item Use ridges to create full transition states with the dimer as normal to a hyperplanar segment
\item Catalytic Selectivity
\item Extend to Quantum HTST
\eit

\placeholder
