\section{Outlook}
\label{sec:summary-outlook}

%\bit
%\item DFT ridges
%\item How ridge calculations interact with different types of saddle points (such as those with $H = \vect{0}$, SPs which are also points of inflection or turning points)
%\item Rigorous performance study.
%\item Discover the limit of how low \sap{2} can be with regards to \sap{1}. i.e. confirm the $5\kB T$ stuff.
%\item Use ridges to create full transition states with the dimer as normal to a hyperplanar segment
%\item Catalytic Selectivity
%\item Extend to Quantum HTST
%\eit

\paragraph{A More Precise Ridge}
A detailed outlook as to further development of the ridge mapping algorithm was presented in \fref{chap:erm}.
Two efforts to reduce corner cutting on the ridge were discussed.
First by implementing a dual tangent scheme in order to minimise instabilities and remove the need for a perpendicular spring force component.
A second tangent estimate is introduced, to which the minimum mode would be perpendicular at all times.
This second tangent would be a more precise representation of the ridge than the previous tangent which is chosen for numerical stability of the MEP part of the algorithm.
The suggested implementation was a central difference one but a more elaborate spline implementation should also be considered.
The second attempt to reduce corner cutting was by iteratively removing the perpendicular component of the spring force.
It is unclear if this second scheme is needed, should the dual tangent scheme be successful.
Both of the schemes were tested on simple systems and showed great promise but further testing and development are needed.

\paragraph{Transition State Theory}
Using the energy ridges to detect anharmonicity is presented in the thesis and a correction factor is even suggested.
However, a more complete transition state rate built from the energy profile of the ridge directly is a certainly a worthwhile topic.
Furthermore, stitching together a transition state from hyperplanar segments, at each image, normal to the minimum mode is a natural progression.
Such a transition state could even be generated in a variational way, where the number of images would serve as the variational parameter.

\paragraph{DFT Ridges}
DFT calculations of ridges were only moderately successful.
Investigating the reasons for this is essential for application of the ridge method in production systems.

\paragraph{Performance}
The performance of the ridge method was not investigated in any detail.
Subsequently, the performance was not optimised beyond the use of parameters that have worked well with the underlying methods.
A full performance study and parameter optimisation should be performed.

\paragraph{Application}
Only a single application was studied in detail.
Further applications are required, in particular, studies of systems that are less known \textit{a priori} to test the usefulness of the intermediate \sap{1} detection.
