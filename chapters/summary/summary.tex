\chapter{Summary}
\label{chap:summary}

\subsubsection{Metal Borohydrides}
The rotational dynamics of two metal borohydrides were investigated in detail with close collaboration to experimental work.
The results of which complemented the experimental results very well.

For $\beta$-\ce{Mg(BH4)2} only rotational events were detected, with roughly $15\%$ of the \ce{BH4} groups activating very low energy ($E_b = 0.03\unit{eV}$) $C_2$-type rotations before the rest activates their $C_2$-type rotations ($E_b = 0.05 - 0.10 \unit{eV}$).
This two-fold activation of the $C_2$-type rotations was found to be dependant on the distance between \ce{B} and its \ce{Mg}-\ce{Mg} axis.
The $C_3$-type rotations were considerably higher in energy ($E_b = 0.13 - 0.30 \unit{eV}$) and did not show the same dependence as the $C_2$-type rotations.

For $\beta$-\ce{Ca(BH4)2}, not only rotational events were detected but also longer range diffusion of hydrogen.
The rotational events have activation energies of $0.09 \unit{eV}$ and $0.15\unit{eV}$ for the $C_3$- and $C_2$-type rotations, respectively.
As for the long range diffusion, many processes were considered but the only one with any significant similarities to the, very scarce, experimental data was that of a \ce{H2}-interstitial.

\subsubsection{Ridge Mapping}
In order to map ridges of functions\footnote{Steepest descent paths between first and second order saddle points.} a method was developed.
By transforming convex subspaces into concave subspaces, the dimer algorithm maps first order saddle points to minima and ridges to minimum energy paths\footnote{or path of least resistance.} (MEP), with regards to the gradient.
Using this transformation to iteratively converge a trial path to the ridge is then achieved using the nudged elastic band (NEB) algorithm for finding minimum energy paths.
Due to numerical instabilities, an artificial method for stabilisation was added to the NEB-type part of the algorithm.
This stabilisation leads to systematic errors in the final path --- only the path, not the second order saddle point ---
Effort to leave out or minimise this stabilisation part of the algorithm were discussed along with a discussion on the reduction of the numerical instabilities introduced by an inaccurate ridge$\rightarrow$MEP mapping.
However, these discussions lacked proper testing due to insufficient time and can, thus, only be considered as an extended and detailed outlooks.

In the context of theoretical reaction rate chemistry, the ridge method was applied to validate the reaction rates offered by the harmonic approximation to transition state theory.
Furthermore, in cases where the harmonic approximation was found lacking, a rough correction factor was offered by calculating the ratio between configurational integrals of the ridge and the harmonic energy profile.

For the \ce{Al} adatom on the \ce{Al}(100) surface system, the lowest energy mechanism for diffusion were investigated, the hop over a ridge site and the non-intuitive concerted motion of several atoms, 2, 3 and 4.
Calculation of the ridges between each pair showed that the harmonic approximation held its ground for the lowest energy mechanism, the concerted motion of 2 atoms, and the hop, while it was found to be severely lacking for the more populous concerted motion mechanisms.
The correction factor for the latter was found to be significant.
It was also found that the ridge between the 3 and 4 atom concerted saddle points on the one hand and the hop saddle point on the other, are separated by the 2 atom concerted saddle point.
In this way finding novel mechanisms is possible using the ridge method, similarly to how the NEB method locates novel minima.

The \ce{Al} system was modelled with the embedded atom method which offers gradients that are clean from numerical noise.
The dimer algorithm requires well behaved gradients since it relies heavily on finite difference methods for estimating derivatives.
In order to test this dependence, DFT calculations on the \ce{SrTiO3} perovskite system with a coupled hydrogen defect were performed.
Due to time constrictions, only a handful of ridge calculations were attempted.
Half of which were successful and found 2 low lying \sap{2}s near processes where a hydrogen defect atom would jump between neighbouring oxygen atoms.
The other half of the ridge calculations were unsuccessful and the path would lie near MEPs instead of ridges.
This behaviour is unexplained and requires further research.
However, there is no reason to suspect the method itself as the culprit as these flaws were not present in the other test systems.

\section{Outlook}
\label{sec:summary-outlook}

\bit
%\item DFT ridges
%\item How ridge calculations interact with different types of saddle points (such as those with $H = \vect{0}$, SPs which are also points of inflection or turning points)
%\item Rigorous performance study.
\item Discover the limit of how low \sap{2} can be with regards to \sap{1}. i.e. confirm the $5\kB T$ stuff.
%\item Use ridges to create full transition states with the dimer as normal to a hyperplanar segment
\item Catalytic Selectivity
\item Extend to Quantum HTST
\eit

\paragraph{A More Precise Ridge}
A detailed outlook as to further development of the ridge detection algorithm was presented in \fref{chap:erm}
Two efforts to reduce corner cutting on the ridge were discussed.
First by introducing a dual tangent scheme in order to minimise instabilities and remove the need for a perpendicular spring force component.
Introducing a second tangent to which the minimum mode would be perpendicular at all times.
This second tangent would be a more precise representation of the ridge than the previous tangent which is focussed at numerical stability for the MEP detection part of the algorithm.
The suggested implementation was a central difference one but a more elaborate spline implementation should also be considered.
The second effort at reducing corner cutting was by iteratively removing the perpendicular component of the spring force.
It is unclear if this second scheme is needed, should the dual tangent scheme be successful.
Both of the schemes were tested on simple systems and showed great promise but further testing and development are needed.

\paragraph{Transition State Theory}
Using the energy ridges to detect non-harmonic transitions is presented in the thesis and a correction factor is even suggested.
However, a more complete transition state rate built from the energy profile of the ridge directly is a certainly a worthwhile topic.
Furthermore, stitching together a transition state from hyperplanar segments, at each image, normal to the minimum mode is a natural progression.
Such a transition state could even be generated in a variational way, where the number of images would serve as the variational parameter in lieu of the transition state itself.

\paragraph{DFT Ridges}
DFT calculations of ridges were only moderately successful.
Investigating the reasons for this \expand \tred{(depends on the Perovskites chapter)}

\paragraph{Performance}
The performance of the ridge method was not investigated in any detail.
Subsequently, the performance was not optimised.
Such studies should be performed.

\paragraph{Application}
Only a single application was studied in detail.
Further applications could include \expand 


