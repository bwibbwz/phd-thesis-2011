\section{Beyond Harmonicity}
\label{sec:beoynd-harmonicity}

The reaction rate offered by the, commonly used, harmonic approximation to transition state theory (\fref{sec:htst}), HTST, assumes two things.
First, that the transition state, $\ts$, can be described with a hyperplane in which the relevant \sap{1} lies and whose normal is the negatively eigenvalued eigenvector at the \sap{1}.
Second, that the energy profile at the $\ts$ and minimum can be approximated with a second degree Taylor expansion.
These assumptions make the the harmonic approximation faster than full TST by orders of magnitude.

The conditions under which HTST is valid are generally that the energy, $E(\vR)$, at the minimum must be sufficiently lower than at the \sap{1} and that the \sap{1} energy must be sufficiently lower than the energy at any neighbouring \sap{2}s.
The literature is unspecific as to what a sufficient difference is but a commonly used value is that $E(\vR_\sap{1}) - E(\vR_\text{min.}) > 5\kB T$~\cite{htst-5ev-2005}.
It is reasonable to assume a similar criterion for $E(\vR_\sap{2}) - E(\vR_\sap{1})$.

Ridge calculations can be used to check if this criterion is fulfilled but further, they can be used to improve the reaction rate.
By comparing the configurational integral of the harmonic energy profile to that of the ridge corrections can be made to the reaction rate.



