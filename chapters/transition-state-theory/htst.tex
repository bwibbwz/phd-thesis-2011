\subsection{Harmonic Transition State Theory}
\label{sec:htst}
To actually calculate the partition functions can be a tedious task and finding a good transition state is a non-trivial one, if possible.
Fortunately, for many systems (e.g. solid state materials), the transition state can be approximated and the configation integrals simplified.

Between any two basins there exists a \sap{1}, which represents the lowest possible potential energy that the system must overcome in order for a tranistion to occur.
Exploiting this, the $\ts$ can a chosen as a hyperplane in which the \sap{1} resides and whose normal is the eigenmode that corresponds with the lowest eigenvalue of the Hessian.
This greatly simplifies the definition of the $\ts$.
Furthermore, in order to simplify the calculation of the partition functions, the PES near both $\ts$, $E_\ts$, and R, $E_\text{R}$, are represented by second degree Taylor expansions,
\beq{htst-taylor-expansion-min}
E_\text{R}(\vect{q}_\text{R}) \approx E(\vR_\text{min.}) + \frac{1}{2} \sum_i^{3N} k_{\text{R},i} q_{\text{R},i}^2,
\eeq
\beq{htst-taylor-expansion-sp}
E_\ts(\vect{q}_\ts) \approx E(\vR_\text{\sap{1}}) + \frac{1}{2} \sum_i^{3N-1} k_{\ts,i} q_{\ts,i}^2,
\eeq
where $\vect{q}_{[\ts, \text{R}]}$ are the eigenvectors of the Hessian, $q_{[\ts, \text{R}],i}$ is the displacement along eigenvector $i$, $k_{[\ts, \text{R}],i}$ are the corresponding Taylor coefficients, all for the $\ts$ and R respectively, and $E(\vR_{[\sap{1}, \text{min.}]})$ are the energies at the \sap{1} and the minimum of the R basin, respectively.
Since the central points of the expansions are stationary, the first derivatives vanish from the expression, leaving only a constants along with the second order terms.

Due to the format of the energy approximations, the configurational integrals become trivial Gaussian integrals,
\beq{htst-gaussian-integral}
\int_{-\infty}^\infty e^{-k_iq_i^2/2 \kB T} dq_i = \sqrt{\frac{2 \pi \kB T}{k_i}}.
\eeq
The reaction rate then reduces to
\beq{htst-reaction-rate-raw}
k_\text{HTST} = \sqrt{\frac{\kB T}{2 \pi \mu}}\frac{e^{-E(\vR_{\sap{1}})/\kB T}\prod_i^{3N-1}\sqrt{\frac{2 \pi \kB T}{k_{\ts, i}}}}{e^{-E(\vR_\text{min.})/\kB T}\prod_i^{3N}\sqrt{\frac{2 \pi \kB T}{k_{\text{min.}, i}}}}.
\eeq
Most of the $\kB T$ and $2\pi$ parameters cancel each other, leaving only one set by means of the different dimensionality of $\ts$ and R.
Rearrangement of \fref{eq:htst-reaction-rate-raw}, along with the multiplication of $\mu^{3N}$ on both sides of the fraction, yields,
\beq{htst-reaction-rate-cleaned}
k_\text{HTST} = (2\pi)^{-1} \frac{\prod_i^{3N} \sqrt{\frac{k_{\text{min.},i}}{\mu}}}{\prod_i^{3N-1} \sqrt{\frac{k_{\ts,i}}{\mu}}}
e^{-(E(\vR_{\sap{1}}) - E(\vR_\text{min})) /\kB T}.
\eeq
It is possible to insert the respective vibrational frequencies,
\beq{htst-vibrational-frequencies}
\nu_\text{*} = \sqrt{\frac{k_\text{*}}{\mu}}(2\pi)^{-1},
\eeq
into \fref{eq:htst-reaction-rate-cleaned},
\beq{htst-reaction-rate}
k_\text{HTST} = \frac{\prod_i^{3N}\nu_{\text{min.},i}}{\prod_i^{3N-1}\nu_{\ts,i}}
e^{-(E(\vR_{\sap{1}}) - E(\vR_\text{min})) /\kB T},
\eeq
for a rate that depends only on the vibartional frequencies and energies at the \sap{1} and minimum as well as the temperature.
This form of the rate agrees with the emperically observed temperature dependance of the rate originally expressed by Arrhenius~\citemiss.
The potential energy barrier is addressed in the exponent while the entropical effects are addressed in the so-called prefactor \expand free energy ...

\bit
\item Discuss which system sit performs well with
\item Discuss possible failures (maybe this belongs in the introduction more than here)
\eit

\section{Common Misconceptions}
\label{tst:misconceptions}

Lingering misconceptions regards some of the concepts in TST \expand

\subsubsection{The "activated complex"}
\placeholder

\subsubsection{Terminology ("Transition State" vs. "Saddle Point")}
Due to the popularity of the harmonic approximation to TST, a common misconception is that saddle points and transition states are interchangeable terms.
This is apparent in the literature, even where HTST is directly discussed or saddle points and reaction paths found.
\tblue{No need to "name and shame"?}

The confusion stems from the fact that the saddle point is the only geometrical feature needed to construct the transition state.
Once found, the saddle point acts as the ... \tblue{Write after the actual definition of HTST has been written out.}

Even though the distinction is not overly important within HTST, it is of central importance in TST, as the actual definition of the transition state is vital.

\incomplete
