\section{Collaboration Between Theoretical and Experimental Work}
\label{sec:cooperation}

This project was built on tight collaboration bewteen experimental and theoretical work.
\bit
\item Idea for investigating certain types of dynamics is born
\item Do the experiments at the appropriate instrument(s)
\item Initial analysis gives good indication as to what type of dynamics have been observed
\item Quickly explain the experimental models used (Chutley-Elliot)
\item Theoretical calculations take these indications and expand on them
\item Theory suggests rates for specific events
\item Experimental calculations are revised (e.g. Different fitting parameters)
\item Any unexplained events get investigated specifically by theory
\eit

\placeholder

\section{Quasielastic Neutron Scattering}
\label{sec:qens}

Neutron scattering is not a direct topic in this thesis but a very short introduction is, nevertheless, in order, since the technique was extensively used in papers \ref{pap:calcium} and \ref{pap:magnesium}.
For a more detailed discussion, see for example \citemiss.

Neutrons are used in the context of hydrogen storage due to deep penetration into the sample and their large interaction with atomic hydrogen.~\tred{[B\'ee]}

An incoming neutron with a wave vector,
\beq{wave-vector}
\bm{k} = \frac{\bm{p}}{\hbar},
\eeq
where $\bm{p}$ is its momentum, interacts with a sample and is scattered to a wave vector, $\bm{k}' = \bm{p}' / \hbar$.
The energy of the neutron is
\beq{neutron-energy}
E = \frac{\hbar^2 |k|^2}{2m} \quad \text{and} \quad E' = \frac{\hbar^2 |k'|^2}{2m}
\eeq
before and after the interaction, respectively, where $m$ is the neutron's mass.
The conservation of momentum allows the definition of a momentum transfer vector, $\vQ = \bm{k}' - \bm{k}$, or scattering vector.
If the wave vectors are not of the same length, before and after the interaction, $\left| \bm{k} \right| \ne \left| \bm{k}' \right|$, energy has been lost/gained and the scattering is referred to as inelastic.
Doppler like broadening of otherwise elastically scattered neutrons is then referred to as quasielastic neutron scattering (QENS).
This broadening is typically caused by diffusive motion within the sample.~\citemiss

\figmiss{QENS spectrum}

A typical neutron scattering spectrum can be seen in \fref{fig:qens-spectrum}
A single spectrum is produced for each $\vQ$ value and should they differ, information about locality of the motion in question can be inferred from the $\vQ$-dependence.
Local motion, such as rotations, can be found anywhere in the sample and is represented by a lack of $\vQ$-dependence.
In particular low $\vQ$ represents large volumes of the sample since $\vQ$ resides in resiprocal space.
The quasielastic part of the spectrum is fitted with one Laurentian for each motion.
From the ratio between the elastic and quasielastic signals at each Q, information about the distance of the motion can be inferred.
Finally, information about the frequency --- or more precisely, the time between motions --- can be inferred from the broadening of the quasielastic signal with $\vQ$.

\bit
\item Look at the articles...
\eit

%The energy window of each instrument is so small that detecting motion of a type different from that under investigation is nearly impossible.

\incomplete
