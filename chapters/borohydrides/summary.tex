\section{Summary}
\label{sec:borohydrides-summary}

Collaboration between experimental work and theoretical work was able to produce a convincing image of the low temperature rotational motion of two borohydrides --- both of which have very high hydrogen capacity ---, a task hindered by the scarcity of appropriate experiments.
Furthermore, low temperature \ce{H2} interstitial diffusion in \ce{Ca(BH4)2} is suspected to occur but the scarcity of experimental data hindered anything beyond reasonable speculations.

When considering systems of events of such a different nature, such as rotational diffusion on the one end and long range diffusion on the other, lead to thoughts on what actually separates the similar events, in this case the rotational diffusion events.
The PESes give some indication as to what sort of \sap{2} lies inbetween but as with the MEPs, the rigid rotation barriers are insufficient.
Thus was born the idea to find the \sap{2} exactly, using a NEB type method, and this effort will be discussed in detail in \fref{chap:erm} and Paper \ref{pap:second-order}.
