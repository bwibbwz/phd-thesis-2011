\section{Introduction}
\label{sec:borohydrides-introduction}


Tejs' comments
\bit
\item Order/Disorder transitions
\item Difficult to get atomic level experimental data
\item Multiple phases
\item Structures are not fully known (but mostly)
\item Large superstructures
\eit

This chapter describes the theoretical work performed for Papers \ref{pap:calcium} and \ref{pap:magnesium} while commenting on the interaction between theoretical and experimental work.

The high hydrogen density, both volumetric and gravimetric, of metal borohydrides makes them interesting research topics as candidates for hydrogen storage.
However, they commonly exhibit poor reversability~\citemiss, slow ab- and de-sorbtion kinetics~\citemiss and too high thermodynamic stability~\citemiss.
\ce{Mg(BH4)2} and \ce{Ca(BH4)2} have lower thermodynamic stability than other borohydrides, e.g. \ce{LiBH4}, while still having high hydrogen denisty ($14.9\%\text{wt}$ and $11.5\%\text{wt}$, respectively)~\citemiss and have been shown partial reversibility~\citemiss.


Using Quasi-Ealstic Neutron Scattering (QENS)

