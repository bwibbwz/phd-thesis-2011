\section{Introduction}
\label{sec:borohydrides-introduction}

% This top bit would be nice as a tagline (i.e. not part of the text itself)
This chapter describes the theoretical work performed for Papers \ref{pap:calcium} and \ref{pap:magnesium} while commenting on the interaction between theoretical and experimental work.

\bit
\item Order/Disorder transitions
\item Difficult to get atomic level experimental data (emphasize), thus, calclations important to help
\item Kinetically limited processes
\item Multiple phases
\item Structures are not fully known (but mostly)
\item Large superstructures
\eit

With high hydrogen density, both volumetric (up to \missing{}~\citemiss) and gravimetric (up to \missing{}~\citemiss), metal borohydrides are of tremendous interest as mobile hydrogen storage candidates.
However, they commonly exhibit poor reversability~\citemiss, slow ab- and de-sorbtion kinetics~\citemiss and too high thermodynamic stability~\citemiss.
Understanding these drawbacks is an important research subject in order to solve them or discover new, better, materials for hydrogen storage.

The work presented here is focussed on detecting and understanding atomic scale hydrogen motion in two species, \ce{Mg(BH4)2} and \ce{Ca(BH4)2}, which have shown lower thermodynamic stability than other borohydrides, e.g. \ce{LiBH4}, while still having high hydrogen denisty ($14.9\%\text{wt}$ and $11.5\%\text{wt}$, respectively)~\citemiss and showing partial reversibility~\citemiss, in order to better understand structural transition and decomposition mechanisms.
Atomic level experiemntal data is scarce and experimental methods are limited to a handful of techniques.
Quasi-Elastic Neutron Scattering (QENS) experiments are sensitive to hydrogen motion on the range of interest and have, to some extent, been used to study interstitial metal hydrides(?) for other species

Collaboration between experimental and theoretical researchers is essential in this regard to understand the ...

Quasi-Ealstic Neutron Scattering (QENS) experiments are sensitive to hydrogen dynamics and have been used to study interstitial metal hydrides~\citemiss.
However, metal borohydrides and other complex hydrides have not recieved as much attention in the literature.

The work presented in Papers \ref{pap:calcium} and \ref{pap:magnesium} focuses on detecting and analysing hydrogen diffusion.

For some of the borohydrides the unitcells are very large and hard to deal with on the theoretical level.
This is well illustrated by the fact that the exact symmetry of the cells are still in active refinement in the literature.~\citemiss

\incomplete
