\section{Introduction}
\label{sec:borohydrides-introduction}
With high hydrogen density, both volumetric and gravimetric, metal borohydrides are of tremendous interest as mobile hydrogen storage candidates.
However, they commonly exhibit poor reversibility, slow ab- and de-sorbtion kinetics and too high thermodynamic stability.~\cite{lithium-stability-2003, borohydride-stability-2006, calcium-stability-2006}
Understanding these drawbacks is an important research subject in order to solve them or discover new, better, materials for hydrogen storage.

The work presented here\footnote{and in full detail in papers \ref{pap:calcium} and \ref{pap:magnesium}} is focussed on detecting and understanding atomic scale hydrogen motion in two species, \ce{Mg(BH4)2} and \ce{Ca(BH4)2} in order to better understand structural transition and decomposition mechanisms.
These have shown lower thermodynamic stability than other borohydrides~\cite{borohydride-stability-2006, calcium-stability-2006}, e.g. \ce{LiBH4}, while still having high hydrogen density ($14.9\%\text{wt}$ and $11.5\%\text{wt}$, respectively) and showing partial reversibility~\cite{magnesium-reversibility-severa-2010, magnesium-reversibility-chong-2011, calcium-reversibility-2007, calcium-reversibility-2008, reversibility-destabilisation-2008}
Out of the possible hydrogen dynamics, rotations of borohydride groups are easily overlooked as uninteresting but they are often associated with order-disorder phase transitions in coordination compounds.~\cite{order-disorder-2006, order-disorder-2010}
Furthermore, longer range diffusion of hydrogen or \ce{BH4} could be part of the decomposition mechanism.
The work presented here is mainly focussed on detecting and understanding the rotational motion of \ce{BH4} groups in the relevant metal borohydrides.
Atomic level experimental data for these types of dynamics is scarce and experimental methods are limited to a handful of techniques, one of which is quasielastic Neutron Scattering (QENS), which is sensitive to hydrogen motion~\cite{qens-bee-1988} on the range of interest.

Due to the scarcity of appropriate experiments, collaboration between experimental and theoretical researchers is essential to fully interpret and understand their results.
For some of the borohydrides the unit cells are very large and hard to deal with on the theoretical level.
This is well illustrated by the fact that the exact structure of the cells is still in active refinement in the literature.~\cite{cabh42-structure-p42m, cabh42-structure-p4}
The existence of multiple phases can complicate matters even further as their local geometry may be similar.~\cite{mgbh42-structure-fddd, mgbh42-phases-2007, mgbh42-phases-2008, mgbh42-phases-2009}
However, for the work presented here, only a single phase was present.
