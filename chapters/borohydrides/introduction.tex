\section{Introduction}
\label{sec:borohydrides-introduction}

%\bit
%\item Order/Disorder transitions
%\item Difficult to get atomic level experimental data (emphasize), thus, calclations important to help
%\item Kinetically limited processes
%\item Multiple phases
%\item Structures are not fully known (but mostly)
%\item Large superstructures
%\item Reading this chapter should give an overview of the theoretical work performed but for a full discussion, refer to the respective papers.
%\eit

With high hydrogen density, both volumetric (up to \missing{}~\citemiss) and gravimetric (up to \missing{}~\citemiss), metal borohydrides are of tremendous interest as mobile hydrogen storage candidates.
However, they commonly exhibit poor reversability~\citemiss, slow ab- and de-sorbtion kinetics~\citemiss and too high thermodynamic stability~\citemiss.
Understanding these drawbacks is an important research subject in order to solve them or discover new, better, materials for hydrogen storage.

The work presented here\footnote{and in full detail in papers \ref{pap:calcium} and \ref{pap:magnesium}} is focussed on detecting and understanding atomic scale hydrogen motion in two species, \ce{Mg(BH4)2} and \ce{Ca(BH4)2} --- which have shown lower thermodynamic stability than other borohydrides~\cite{borohydride-stability-2006, calcium-stability-2006}, e.g. \ce{LiBH4}, while still having high hydrogen denisty ($14.9\%\text{wt}$ and $11.5\%\text{wt}$, respectively) and showing partial reversibility~\citemiss --- in order to better understand structural transition and decomposition mechanisms.
Out of the possible hydrogen dynamics, rotations of borohydride groups are easily overlooked as uninteresting but they are often associated with order-disorder phase transitions in coordination compounds.~\cite{order-disorder-2006, order-disorder-2010}
Furthermore, longer range diffusion of hydrogen or \ce{BH4} could be part of the decomposition mechanism.
Atomic level experimental data for these types of dynamics is scarce and experimental methods are limited to a handful of techniques, one of which is quasielastic Neutron Scattering (QENS), which is sensitive to hydrogen motion~\cite{qens-bee-1988} on the range of interest and have, to some extent, been used to study interstitial metal hydrides for other species.~\citemiss

Due to the scarcity of appropriate experiments, collaboration between experimental and theoretical researchers is essential to fully interpret and understand their results.
For some of the borohydrides the unitcells are very large and hard to deal with on the theoretical level.
This is well illustrated by the fact that the exact structure of the cells is still in active refinement in the literature.~\cite{cabh42-structure-p42m, cabh42-structure-p4}
The existance of multiple phases can complicate matters even further as their local geometry may be similar.~\citemiss [32-38 in paper \ref{pap:magnesium}]
However, for the work presented here only a single phase was present.

%Metal borohydrides and other complex hydrides have not recieved as much attention in the literature.
%\expand


