\section{Calcium Borohydride [$\beta$-\ce{Ca(BH4)2}]}
\label{sec:borohydrides-calcium}

\bit
\item Calculate rigid rotations of \ce{BH4-} to get a "feel" for the landscape
\item Then calculate the reaction paths (MEPs) and saddle points
\item Prepare HTST rates and compare with experiments
\item Revise the experimental rates as applicable
\item Do further MEP calculations for mechanisms that still remain unexplained in the experimental data
\eit

The calculational parameters can be found in Paper \ref{pap:calcium}.

The experimental data suggested 3 separate processes, two of which were assigned to rotational diffusion of hydrogen, while the latter was assigned to longer range diffusion of hydrogen.

\subsubsection{Rotation of \ce{BH4-}}
The bulk of the initial data from the QENS experiments indicated rotational diffusion of hydrogen, while possible high temperature long-range diffusion was also detected.
Thus, at first, only rotations around the possible symmetry axes, $C_2$ and $C_3$, of the \ce{BH4-} unit were considered.
Contour plots of the rigid rotation can be produced in order to get an overview of the possible rotations, pure or coupled, and their relative barrier heights.
\figmiss{All the PESes. Perhaps there is only one for \ce{Ca(BH4)2}?}
The contour plots give an important insight as to which of the possible rotations are the interesting ones and, thus, calculating the more resource intensive MEPs can be limited to the interesting processes.
Furthermore, due to slight breaking of the symmetry certain rotations will have different barriers \expand

\Fref{fig:ca-pes01} shows that a $C_3$ axis is lowest in energy and that a wobbly $C_2$ axis with a small intermediate minimum is the lowest energy $C_2$ type rotation.
Using these as the starting paths for NEB calculations, the MEPs shown overlayed in \fref{fig:ca-pes01} are found and shown as energy profiles in \fref{fig:ca-barriers}.

These barriers complement the experimental data very well, both with regards to the barrier height as well as the charactristic times (?) which are derived from the HTST reaction rates. (see Table 2 in Paper \ref{pap:calcium})

\subsubsection{Long-range diffusion \pending}

The long range diffusion was based on only two data points

\incomplete

