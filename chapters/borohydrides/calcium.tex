\section{Calcium Borohydride [$\beta$-\ce{Ca(BH4)2}]}
\label{sec:borohydrides-calcium}

\bit
\item Prepare HTST rates and compare with experiments
\item Revise the experimental rates as applicable
\item Do further MEP calculations for mechanisms that still remain unexplained in the experimental data
\eit

The calculational parameters can be found in DFT Calculations section of Paper \ref{pap:calcium}.

The experimental data suggested three separate processes, two of which were assigned to rotational diffusion of hydrogen, while the latter was assigned to longer range diffusion of hydrogen.

\subsubsection{Rotation of \ce{BH4-}}
The bulk of the initial data from the QENS experiments indicated rotational diffusion of hydrogen, while possible high temperature long-range diffusion was also detected.
Thus, the work focused mainly on rotations around the possible symmetry axes, $C_2$ and $C_3$, of the \ce{BH4-} unit.
By choosing the appropriate axes, a contour plot of the rigid rotation was produced in order to get an overview of the possible rotations, pure or coupled, and an estimate of their relative barrier heights.

 \figmiss{The PES and energy profiles, side-by-side}

Since the rotational system has three dimensions but the contourplot has only two, the axes had to be chosen very carefully as to avoid missing an interesting event.
Due to the alignment of the  threefold symmetry of the \ce{BH4-} unit's environment with the unit's $C_2$ axes, all the $C_2$ axes are roughly equal, so the axis was chosen to maximise the hydrogen-calcium distance but this choice was not critical.
The choice of $C_3$ axis was not important either as they all get implicitly sampled via the environmental symmetry.
Thus, in the end the particular choice of axes was not important due to symmetry reasons.

The contour plots give an important insight as to which of the possible rotations are the interesting ones and, thus, calculating the more resource intensive MEPs can be limited to the interesting processes.

Some symmetry breaking can be seen in the PESes.
This is to be expected as the tetragonal symmetry can be slightly broken during the minimisation.

\Fref{fig:ca-pes01} shows that a $C_3$ axis is lowest in energy and that a wobbly $C_2$ axis with a small intermediate minimum is the lowest energy $C_2$ type rotation.
Using these as the starting paths for NEB calculations, the MEPs shown overlayed in \fref{fig:ca-pes01} are found and shown as energy profiles in \fref{fig:ca-barriers}.

Right away, the theoretical data was in good agreement with the experimental data, nevertheless, further refinement of the experimental data was possible after \expand
This beautifully illustrates how theory and experiments complement and can be used to benefit each other.

In the end, the calculations complemented the experimental data very well, both with regards to the barrier height as well as the charactristic times which are derived from the HTST reaction rates. (see Table 2 in Paper \ref{pap:calcium})

\subsubsection{Long-range diffusion \pending}

\bit
\item Theory suggests not pure hydrogen, thus, probably solvent
\item Cite "The crystal structure of the first borohydride borate, Ca3(BD4)3(BO3)" by M. D. Riktor
\eit

The long range diffusion was based on only two data points.

\incomplete

