\section{Magnesium Borohydride [$\beta-$\ce{Mg(BH4)2}]}
\label{sec:borohydrides-magnesium}

\bit
\item Calculate rigid rotations of \ce{BH4-} to get a "feel" for the landscape for all the different borons in the cell
\item Show multiple PESes
\item Then calculate the reaction paths (MEPs) and saddle points
\item Prepare HTST rates and compare with experiments
\item Revise the experimental rates as applicable
\item Flat landscape $\rightarrow$ Introduce the next section
\eit

The calculational supercell was relaxed, from the  $fddd$ spacegroup ($\#70$), totalling 176 atoms.

The structure consists of 5 symmetry inequivelant \ce{BH4} sites.
For each one a PES was constructed

\figmiss{All the PESes and possibly the barriers}

\figmiss{Barrier dependancy on distance

Each \ce{BH4} unit is wedged between, or close to an axis at a distance of $L$, the two nearest neighbour magnesium atoms. \expand

\subsubsection{Rotation of \ce{BH4-}}
The experimental data suggested only rotational dynamics present at the relevant temperatures.
Thus, the work exclusively revolved around understanding which rotations correpsonded with the experiments.

For each symmetry inequivelant \ce{BH4} unit, a contourplot was constructed in order to get an overview of the possible rotations.
Since the rotational system has three dimensions but the contourplot has only two, the axes had to be chosen carefully as to sample all interesting events for each site.
Choosing the $C_2$ axis essentially consisted of 


\placeholder
