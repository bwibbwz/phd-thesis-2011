\subsection{Nudged Elastic Band}
\label{sec:neb}

Finding Steepest Decent Paths (SDPs) from a given point is simple by following the gradient with a small step size.
On the other hand, finding specific SDPs that end at minima is not.
The Nudged Elastic Band (NEB) algorithm is concerned with aligning a path with certain SDP paths, the MEP, leading to two minima from a common \sap{1}.

An initial guess of the path is commonly a, discretisised, linear interpolation bewteen the minima but any guess is suitable as long as the force can be calculated.
Each of discretisation points is referred to as an image and is simply a replica of the system in question but with different coordinates from the other.
Each image feels two separate forces.
First there is the real force, coming from the PES.
However, only the component perpendicular to the path is retained,
\beq{neb-real-force}
\vF^\perp = \vF - (\vF \cdot \uvt) \uvt,
\eeq
where $\uvt$ is the tangent to the path.
Second, there is a force acting purely along the tangent, which is tasked with equally spacing the images along the path.
There are multiple ways to implement this force but a succesful way is to depend on the norms to the neighboring images instead of the full vectors~\cite{neb-tangent-2000},
\beq{neb-spring-force}
\vF^\text{S} = \text{\missing}.
\eeq
Combining these two forces into an effective force,
\beq{neb-effective-force}
\vF^\text{eff} = \vF^\perp + \vF^\text{S},
\eeq
will iteratively bring the path to a discretisised version of the MEP.

\subsubsection{Tangent}

Since the path is discretisised, the tangent is not obviously defined.
Multiple possibilities for defining the tangent are available but one that considers only the displacement to the neighboring higher energy image has been succesful in minimising "kinks" in the paths.~\cite{neb-tangent-2000}

For a given image, numbered $i$, check its energy in relation with the neighboring images.
It will into one of four cases:
\ben{neb-tangent-cases}
\item $E_{i-1} < E_i < E_{e+1}$
\item $E_{i-1} > E_i > E_{e+1}$
\item $E_{i-1} > E_i < E_{e+1}$
\item $E_{i-1} < E_i > E_{e+1}$
\een
The first two cases yield simple definitions of the tangent,
\beq{neb-tangent-plus}
\uvt = \missing \quad \text{if} \quad E_{i-1} < E_i < E_{e+1}
\eeq
and
\beq{neb-tangent-minus}
\uvt = \missing \quad \text{if} \quad E_{i-1} > E_i > E_{e+1}.
\eeq
The other two cases occur when the image is either a local maximum or a local minimum.
In these latter cases a linear combination of the possible tangents are used, in order to avoid any discontinous artifacts.


\subsubsection{Finding the Exact Saddle Point}
The forces described above do not guarantee that an image will exactly find the \sap{1} in question.
By decoupling the top energy image from the spring force it becomes independant.
... similar to the dimer force scheme ...

\subsubsection{Other Similar methods}
\tblue{Is this a section we'd like, where we could cite the various methods and reviews?}

\incomplete
