\section{Introduction}
\label{sec:methods-introduction}

% ------------------------------------------------------------------
\bit
\item Perhaps use the Newton/Schr\"odinger figure I made a bit back (modified if needed)
\item General on atomistic simulations (seguing to potential functions and DFT)
\item Why statistical methods
\item Why path techniques (introduction to be expanded on in the TST section)
\item It is now "possible" to do long timescale MD simlations but statistical methods are still more suited to find "all" the processes (chat with Elvar on MD)
\item (Why these methods but not some other?)

\item Would be nice to have a figure of each of the main guys for each methodology
\bit
\item Born, Oppenheimer
\item Hesse
\item Arrhenius, Kramer
\item Hannes, Graeme
\item Hohnberg, Kohn
\eit
\eit

The order of the topics for discussion in this chapter is still a bit off.
Born-Oppenheimer is needed before TST but it should be introduced at the same (similar) time as the Schr\"odinger equation which in turn should accompany the DFT section (close to which the potential function section should lie).

\placeholder
