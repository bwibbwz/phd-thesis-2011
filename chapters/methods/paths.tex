\subsection{Steepest Descent Paths}
\label{sec:sdps}

Steepest Descent Paths (SDPs) are paths for which there is no perpendicular gradient component.
They can be created by following the negative gradient until it vanishes.
SDPs can begin anywhere, apart from points with a vanishing gradient, but, generally, end at minima (most common) or \sap{}s.
Two sorts of SDPs are of particular interest for the work presented in this thesis, both of which are discussed below.

% ------------------------------------------------------------------
\subsubsection{Minimum Energy Paths}
\label{sec:meps}

\bit
\item Need a good definition and/or citation.
\item A reaction path
\eit

Check this reference:
Transition-Path Theory and Path-Finding Algorithms for the Study of Rare Events
(Weinan E and Eric Vanden-Eijnden)


\placeholder

% ------------------------------------------------------------------
\subsubsection{Ridges}
\label{sec:ridges}

\bit
\item Need a good definition and/or citation.
\item Special steepest decent paths, which do not end at minima but saddle points.
\eit

\placeholder


