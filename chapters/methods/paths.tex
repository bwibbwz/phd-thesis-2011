\section{Steepest Descent Paths}
\label{sec:sdps}

\figmiss{Showing a few general SDPs and at least one MEP and one ridge}

Steepest Descent Paths (SDPs) are paths on a multidimensional function, $f$, for which there is no perpendicular gradient component,
\beq{sdp-defintion}
\nabla f - (\nabla f \cdot \tau)\tau = \vect{0},
\eeq
where $\tau$ is the tangent to the path.
They can be created by following the negative gradient until it vanishes.
SDPs can begin anywhere, apart from points with a vanishing gradient, but, generally, end at minima (most common) or \sap{}s.
Two sorts of SDPs are of particular interest for the work presented in this thesis, both of which are discussed below.

\bit
\item Missing citation, maybe it is okay to use \cite{gradient-extramals-ruedenberg-1993}
\eit

\subsection{Minimum Energy Paths}
\label{sec:meps}

\bit
\item Need a good definition and/or citation.
%Check this reference:
%Transition-Path Theory and Path-Finding Algorithms for the Study of Rare Events
%(Weinan E and Eric Vanden-Eijnden)
\eit

The Minimum Energy Path (MEP) is specific SDP, used in the context of theoretical reaction chemistry and represents a likely reaction path on the PES.
The MEP's formal definition is generally the same as for any SDP (\fref{eq:sdp-definition}) but with the added criteria that the energy must be at a minimum perpendicular to the path~\cite{neb-polemic-henkelman1},
\beq{mep-definition}
\text{\expand}
\eeq

\subsection{Ridges}
\label{sec:ridges}

Similar to MEPs, ridges are specific SDPs, except with beginning and end points in \sap{2}s and \sap{1}s respectively.
Of course, MEPs do not exists near \sap{2}s, and the anlogue ends there as the criteria for ridges \expand

\bit
\item Need a good definition and/or citation.
\eit

