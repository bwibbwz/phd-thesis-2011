\section{First Order Saddle Point Methods}
\label{sec:sps}

Finding saddle points is a non-trivial task in multiple dimensions, when only local information is available.

\incomplete

% ------------------------------------------------------------------
\subsection{Nudged Elastic Band}
\label{sec:neb}

Finding Steepest Decent Paths (SDPs) from a given point is simple by following the gradient with a small stepsize.
On the other hand, finding specific SDPs that end at minima is not.
The goal here is to find two SDPs each leading from the same \sap1 to different minima without any further information than the minima.


\incomplete

% ------------------------------------------------------------------
\subsection{Dimer}
\label{sec:dimer}

Given only an initial point, $\vR$, on a multidimensional function, $F(\vR)$, the goal is to, iteratively, locate a nearby \sap1, using no direct calculation of the Hessian, i.e. using only functional values and the first derivatives.
Indiredct information about the Hessian is, however, used in the form of an estimate of the eigenmode corresponding to its lowest eigenvalue (minimum mode).
Using the minimum mode, $\uvn$, it is possible to locally transform \sap1s to minima and using conventianl techniques to find them.

The dimer method can be split into two independant phases.
\ben{dimer-phases}
\item Estimating the minimum mode.
\item Transforming the gradient to make \sap1 seem as minima.
%\item Invert the gradient components that lie parallel to the minimum mode and translating system.
\een

\subsubsection{Minimum Mode Estimation}
A pair of points (the dimer), $[\vR_\text{A}, \vR_\text{B}]$, are chosen, close to current point $\vR_0$, such that
\beq{dimer-separation}
\vR_\text{A} + \partial_\text{sep}\uvn = \vR_0 \quad \text{and} \quad \vR_\text{B} - \partial_\text{sep}\uvn = \vR_0
\eeq

\subsubsection{Gradient Transformation}
Once a minimum mode estimate is available for the current point, $\vR_0$, it is possible to transform the gradient so that any \sap1 is transformed to a minimum.
\beq{dimer-transform}
\nabla{}F(\vR_0)
\eeq

\tblue{This latter step is not unique to the dimer algorithm... Lancoszs...}



\incomplete
