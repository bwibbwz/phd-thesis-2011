\section{Transition State Theory}
\label{sec:tst}
% ------------------------------------------------------------------

Transition State Theory (TST) is a formalism dedicated to providing reaction rates in a purely theoretical setting where running dynamical trajectories of the required length is infeasible.

\subsubsection{The Timescale Problem}
In atomistic simulations there is a often a clear separation of timescales between the most interesting events.
With local vibration of the atoms being very fast, typically on the order of \missing,
while non-local reordering of atoms, or reactions?, can be considered as rare-events, typically occur on the order of \missing.

This separation makes it nearly impossible to perform dynamical trajectories that include both types of events.
In fact, such calculations would generally take a time on the order of \missing.

\figmiss{Timescale problem ()I have one somewhere, just need to polish it and stick it in}


\placeholder
