\section{Transition State Theory}
\label{sec:tst}
% ------------------------------------------------------------------

Transition State Theory (TST) is a formalism dedicated to providing reaction rates in a purely theoretical setting where running dynamical trajectories of the required length is infeasible.

\recent

\incomplete

\subsubsection{The Time-scale Problem}
\bit
\item Don't undersell current long time MD simulations
\eit
\figmiss{Timescale problem (I have one somewhere, just need to polish it and stick it in)}
In atomistic simulations there is a often a clear separation of timescales between the most interesting events.
With local vibration of the atoms being very fast, typically on the order of \missing,
while non-local reordering of atoms, or reactions?, can be considered as rare-events, typically occur on the order of \missing.

%This separation makes it nearly impossible to perform dynamical trajectories that include both types of events.
%In fact, such calculations would generally take a time on the order of \missing.
Even though performing dynamics long enough to accurately describe both types of events, is technically possible~\citemiss, the key to long-timescale dynamics still lies outside the reach of such methods as they spend most of their time performing dynamics that are not of particular interest.

In the problem lies also the solution ...

\recent

\incomplete

\subsubsection{Statistical Methods}

\bit
\item Configuration Integrals
\eit

Take advantage of the time-scale separation ...

TST aims at overcoming the time-scale problem by way of "averaging" over the fast vibrational motion to present a reaction rate for the rare-events.

Boltzmann distribution...
Maxwell distribution (velocity)...

\incomplete

\subsubsection{Assumptions}
\bit
\item Born-Oppenheimer
\item Classical Dynamics (No tunnelling)
\item Thermal Equilibrium (Boltzmann)
\item No recrossings (Most severe and the reason for the variational property of the TS???)
\eit
\placeholder

\subsubsection{Reaction Rates}
\placeholder

\subsubsection{Variational TST}
\bit
\item Dynamical Recrossings
\eit

\subsubsection{Quantum TST}
\bit
\item If a short introduction with citations is possible.
\item No long discussion on QTST.
\eit

\subsubsection{Misconception???}
A prevailant misconception regarding TST is that there is an "activated complex" that every rare-event becomes before moving to the product state

\placeholder

