\section{Transition State Theory}
\label{sec:tst}
% ------------------------------------------------------------------

Transition State Theory (TST) is a formalism dedicated to providing reaction rates in a purely theoretical setting where running dynamical trajectories of the required length is infeasible.

\expand

\subsection{Motivation}

\subsubsection{The Time-scale Problem}

\figmiss{Timescale problem (I have one somewhere, just need to polish it and stick it in)}

In atomistic simulations there is a often a clear separation of timescales between the most interesting events.
With local vibration of the atoms being very fast, typically on the order of $10^{14} \unit{Hz}$~\citemiss,
while non-local reordering of atoms, motion between minima on the PES, often considered as rare-events, typically occur on the order of $10^3 \unit{Hz}$ for a system with an energy barrier of $0.5 \unit{eV}$.~\citemiss
Depending on the temperature of the system, a tremendous amount of calculations (on the order of $10^10$) would have to be performed to have a reasonable probability of seeing each rare-event while also fully modelling the details of the atomic vibrations.

Even though performing dynamics long enough to accurately describe both types of events, is technically possible~\citemiss, e.g. by minimising vibrations to allow for longer timesteps~\citemiss or modifying the PES~\cite{hyperdynamics-voter-1997}, the key to long-timescale dynamics still lies outside the reach of such methods as they spend most of their time performing dynamics that are not of particular interest for molecular dynamics.
This is true, in particular, for calculations that rely on costly, but accurate, quantum calculations and/or large systems.

In the problem lies also the solution, when there is such a clear separation of time-scales, it is possible to treat the problem with statistical methods.
Essentially dealing with the vibrational areas of the PES (the basins) by averaging over them and seperately locating ares of transitions.

\subsection{The Basics}

\subsubsection{Assumptions}
There are four main assumptions made within TST, which vary in severity.
\paragraph{Born-Oppenheimer}
The Hamiltonian of the system can be separated, resulting in a single PES for the trajectories.
\paragraph{Classical Dynamics}
No tunnelling or other quantum effects --- apart from those offered by the calculational method of the PES --- are taken into account in the vanilla TST.
The motion of the system is governed by classical dynamics.
Quatnum effects are included in extensions to TST (see for example \cite{qtst-hj-1997, qtst-hj-1998, qtst-hj-2009}~\citemiss) but this is not relevant to the work presented in this thesis and will not be discussed further.
\paragraph{Thermal Equilibrium}
The system will spend considerable time in each basin of the PES, long enough such that the Boltzmann distribution for ... can be employed.
For systems with large energy barriers in comparison with the thermal energy~\footnote{A rule of thumb is that $E_\text{b} > 5\kB T$~\citemiss}, this assumption holds well but for smaller barriers or high temperatures it breaks down as thermal equilibrium becomes difficult to achieve and the separation of timescales is lost
\paragraph{No Re-Crossings}
Once the system has left a basin, it does not return for a significant amount of time (long enough to thermalise in the new basin).
This assumption is the most serious and efforts to minimise its effects are discussed below.

\subsubsection{The Transition State}

\figmiss{Transition State}

When dividing the phase space (position and momentum) into two areas, reactants (R) and products (P), a hypersurface separating the two emerges.
This hypersurface is referred to as the transition state ($\ts$)\footnote{The dagger, $\ts$, is a conventional symbol for the transition state} and represents a bottleneck region which each reactive trajectory crosses on its way from R to P.
Commonly, momentum is disregarded in the definition of the phase space, which is then  $3N$ dimensional, within which the $\ts$ is a $3N-1$ dimensional subspace.

\subsubsection{Statistical Methods}

\bit
\item Configuration Integrals (or partition functions?)
\eit

Take advantage of the time-scale separation ...

TST aims at overcoming the time-scale problem by way of "averaging" over the fast vibrational motion to present a reaction rate for the rare-events.

Boltzmann distribution...
Maxwell distribution (velocity)...

%The partition functions require the total energy of the $N$ atoms system,
%\beq{tst-total-energy}
%E(\vR, \vect{v}) = V(\vR) + \frac{1}{2} \sum_i^{N} m_i \left| \vect{v}_i \right|^2
%\eeq
%with atoms at $\vR$ with velocity $\vect{v}$, travelling on the PES, $V$.

\incomplete

The most important assumption remains, any trajectories that cross the TS do not recross it.
This assumtion is also the weakest link \expand

\subsection{Reaction Rates}
The reaction rate, $k^\text{TST}$, is defined as the probability of being at the $\ts$,
%$P_\ts = \langle \delta(\vR - \vR_\ts) \rangle_\text{R}$,
\beq{ts-probability-partition-simple}
P_\ts = \frac{Z_\ts}{Z_\text{R}},
\eeq
thermally averaged over R, with velocity away from R,\citemiss
\beq{tst-rate-initial}
k^\text{TST} = \frac{1}{2}|\vect{v}_\perp| P_\ts,
\eeq
where $\vect{v}_\perp$ is the velocity component perpendicular to the $\ts$ and the factor $1/2$ comes from the fact that only half of the trajectories will be moving away from R.
Due to the different dimensionality of the $\ts$ and R, $P_\ts$ isn't strictly a probability as it has the unit $\unit{m^{-1}}$. \expand

The velocity at each point $\vR$ can be taken from a Maxwell distribution, \tred{(Why is this okay?)}
\beq{tst-maxwell-velocity}
\langle |\vect{v}| \rangle = \sqrt{\frac{2kT}{\pi m}},
\eeq
where $m$ is the mass of the particles \tred{(This is wrong)}.
The rate then becomes
\beq{tst-rate-full}
k_\text{TST} = \sqrt{\frac{kT}{2\pi m}} \frac{Z_\ts}{Z_\text{R}}. \quad \text{\expand}
\eeq

\placeholder

\subsubsection{Optimisation of the Rate}
\bit
\item Dynamical Recrossings
\item Variational TST
\item Search for the true TS
\eit

Choosing a good $\ts$ is essential for the accuracy of TST.

%The choice of the TS is essential for an accurate rate.
In fact, due to the assumption of no recrossings, TST offers a variational way to discover the true rate,
\beq{tst-variational-rate}
k^\text{TST} \ge k^\text{exact},
\eeq
where the transition state is the variational parameter.

%\subsubsection{Quantum TST}
%\bit
%\item If a short introduction with citations is possible.
%\item No long discussion on QTST.
%\eit

\subsubsection{Assumptions Revisited}
Discuss the implications of the assumptions.
\bit
\item Born-Oppenheimer
\item Classical Dynamics (No tunnelling)
\item Thermal Equilibrium (Boltzmann)
\item No recrossings (Most severe and the reason for the variational property of the TS???)
\eit
\placeholder

%\subsubsection{Misconception???}
%A prevailant misconception regarding TST is that there is an "activated complex" that every rare-event becomes before moving to the product state
%
%\placeholder

\subsection{Methods for Metadynamics}
\bit
\item Some references here would be nice but \tblue{not a priority}.
\item Hyperdynamics
\item AKMC
\eit
\placeholder

