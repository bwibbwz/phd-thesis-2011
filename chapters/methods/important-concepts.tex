\section{Important Concepts}
\label{sec:important-concepts}

This secion shortly introduces a few concepts that are important to the work being presented.

\incomplete
% ------------------------------------------------------------------
\subsection{The Born-Oppenheimer Approximation}
\label{sec:born-oppenheimer}

In order to simplify calculation of large atomic systems, the difference in weight of the electrons and the nuclei is exploited by performing tha calculations in two steps.
First the electronic wavefunction is determined while the nuclei are kept fixed followed by a calculation for the motion of the nuclei while the electronic wavefunction is kept fixed.

\bit
\item Introduce PES
\eit

\incomplete

% ------------------------------------------------------------------
\subsection{Eigenmodes / Eigenvalues}
\label{sec:eigenmodes}

\placeholder

% ------------------------------------------------------------------
\subsection{Reduced Vector-space}
\label{sec:reduced-space}

\placeholder

% ------------------------------------------------------------------
\subsection{The Hessian Matrix}
\label{sec:hessian}

For a real function $f$ of $n$ variables, $\vect{x} = (x_1, x_2, \ldots, x_n)$,
there exists an $n\times n$ matrix, $H$, which contains all the second partial derivatives, if they exist,
\beq{hessian-matrix}
H =
\begin{bmatrix}
\vspace{0.5em} % To create a bit of space AFTER the first line.
\frac{\partial^2f}{\partial x_1^2} &
\frac{\partial^2f}{\partial x_1 \partial x_2} &
\cdots &
\frac{\partial^2f}{\partial x_1 \partial x_n} \\

\frac{\partial^2f}{\partial x_2 \partial x_1} &
\frac{\partial^2f}{\partial x_2^2} & 
\cdots &
\frac{\partial^2f}{\partial x_2 \partial x_n} \\

\vdots & \vdots & \ddots & \vdots \\

\frac{\partial^2f}{\partial x_n \partial x_1} &
\frac{\partial^2f}{\partial x_n \partial x_2} &
\cdots &
\frac{\partial^2f}{\partial x_n^2} &
\end{bmatrix}
\eeq
The second derivative of a function represents, in particular, information about its local curvature, or how rapidly the first derivative changes.
$H$ is named after the German mathematician Ludwig Otto Hesse and is commonly referred to as the Hessian matrix, or Hessian for short. \tred{[original ref]}

In the context of atomic simulations $n$ is generally 3 times the number of atoms in the system, as each one has 3 independant degrees of freedom.
The function in question is often the potential energy of the system, whose negative gradient is commonly referred to as the force.
In most modern software packages both the potential energy and force are readily available while the second derivatives are, generally, not available without explicit and, often, costly calculations.

\recent

% ------------------------------------------------------------------
\subsection{Saddle Points}
\label{sec:sps}

Saddle points are stationary points, i.e. with zero gradient, on multidimensional function, $f(\vR)$, that are neither maxima nor minima.

The most common image of a saddle (point) is the function $f(x, y) = x^2 - y^2$ which near $(x,y) = (0,0)$ resembles a saddle, used when riding horses (se figure...), curving upwards in one direction and downwards in the other.
\figmiss{Comparison of a saddle points environment and a saddle used on a horse.}
This most common image of a saddle point lacks a few elements to be to tell their whole story.

On functions of higher dimensionality than $2$, different orders of saddle points are possible.
The order of the saddle point is decided by the amount of directions that are at a maximum, rather than a minimum.
As such, figure ... show a first order saddle point on a two dimensional function.

\bit
\item What about 1D saddles (e.g. $f(x = 0) = x^3$)
\item What about similar constructs in more dimensions (e.g. like the above in one direction but a maximum in the other)
\eit

\recent

\incomplete

% ------------------------------------------------------------------
\subsection{Steepest Decent Paths}
\label{sec:sdps}


\placeholder

Two sorts of SDPs are of particular interest for the work carried out here and are discussed below.

% ------------------------------------------------------------------
\subsubsection{Minimum Energy Paths}
\label{sec:meps}

\placeholder

% ------------------------------------------------------------------
\subsubsection{Ridges}
\label{sec:ridges}

Special steepest decent paths, which do not end at minima but saddle points.

\placeholder


