\subsection{Saddle Points}
\label{sec:sps}

Saddle points are stationary points, i.e. with zero gradient, on a multidimensional function, $f(\vR)$, that are neither maxima nor minima.

A common view of a saddle (point) is the function $f(x, y) = x^2 - y^2$ which near $(x,y) = (0,0)$ resembles a saddle, used when riding horses (se figure...), curving upwards in one direction and downwards in the other.
\figmiss{Comparison of a saddle points environment and a saddle used on a horse.}
This most common image of a saddle point lacks a few elements to be to tell their whole story.
On functions of higher dimensionality than $2$, different orders of saddle points are possible.
The order of the saddle point is decided by the amount of directions that are at a maximum, rather than a minimum.
As such, figure ... show a first order saddle point on a two dimensional function.

\bit
\item What about 1D saddles (e.g. $f(x = 0) = x^3$)
\item What about similar constructs in more dimensions (e.g. like the above in one direction but a maximum in the other)
\eit

\recent

\incomplete
