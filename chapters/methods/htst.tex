\subsection{Harmonic Transition State Theory}
\label{sec:htst}

% ------------------------------------------------------------------

\placeholder

\subsubsection{Terminology ("Transition State" vs. "Saddle Point")}
Due to the popularity of the harmonic approximation to TST, a common misconception is that saddle points and transition states are interchangeable terms.
This is apparent in the literature, even where HTST is directly discussed or saddle points and reaction paths found.
\tblue{No need to "name and shame"?}

The confusion stems from the fact that the saddle point is the only geometrical feature needed to construct the transition state.
Once found, the saddle point acts as the ... \tblue{Write after the actual definition of HTST has been written out.}

Even though the distinction is not overly important within HTST, it is of central importance in TST, as the actual definition of the transition state is vital.

\recent

\incomplete
