% ------------------------------------------------------------------
\subsection{The Born-Oppenheimer Approximation}
\label{sec:born-oppenheimer}

In order to simplify --- and in many cases, make possible --- quantum calculations of large atomic systems, the difference in weight of the electrons and the nuclei is exploited by performing the calculations in two steps.
First the electronic wavefunction is determined while the nuclei are kept fixed followed by a calculation for the motion of the nuclei while the electronic wavefunction is kept fixed.
The assumption is, essentially, that for any motion of the nucleii, the electrons will move instantly to accomodate.

For many systems this is a good approximation, however, ...

\bit
\item Introduce PES
\item Original citation~\cite{born-oppenheimer-1927}
\item "A good historical source is J. O. Hirschfelder, C. F. Curtiss, and R. B. Bird, Molecular Theory of Gases and Liquids, John Wiley and Sons, New York (1954)."\footnote{taken from \url{http://simons.hec.utah.edu/TheoryPage/lit_art.htm\#born}}
\eit

\incomplete
