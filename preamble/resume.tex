% Rename the Icelandic abstract
%\renewcommand{\abstractname}{\'Agrip}
%\begin{abstract}
%\TH{}etta er frekar magna\dh{}.
 
%\end{abstract}
% Restore the abstract to English

%\begin{center}
%\section*{}
%\vspace{20mm}
\chapter*{Resum\'e}
\addcontentsline{toc}{chapter}{Resum\'e}

%\selectlanguage{danish}

Forståelse af reaktionshastigheder er en essentiel del af moderne forskning på atomar skala.
I denne afhandling beskrives anvendelsen af veletablerede metoder til beskrivelse af vigtige hydrogenlagrings-systemer. Herefter videreudvikles metoderne for at kunne opnå en bedre beskrivelse af reaktionsvejsomgivelserne, for herigennem at bestemme reaktionshastigheden med større nøjagtighed.

Komplekse borohydrider er materialer med en høj hydrogenlagrings-kapacitet og er yderligere meget termodynamisk stabile (for stabile til hydrogen lagring).
For at få bedre indsigt i de strukturelle overgange i to sådanne materialer, \ce{Ca(BH4)2} og \ce{Mg(BH4)2}, blev der udført lav-temperatur rotationsdynamik eksperimenter.
%(I am actually not sure what you mean excactly when talking about structural transitions, but I think this translation should suit most of it :-) furthermore I am not sure whether "rotationsdynamik" is the correct word to use - I do not know much (anything) about it.. so you should probably ask an expert :-) )
Arbejdet der præsenteres her omhandler assistance i forbindelse med dataanalyse, ved hjælp af tæthedsfunktionalteori-beregninger, på de mulige dynamiske hændelser.
For \ce{Mg(BH4)2}, var der god overensstemmelse med eksperimenter. $C_2$-type rotationer observeres ved lavere temperaturer end $C_3$-type rotationer og cirka $15\%$ af \ce{BH4}-enhederne aktiveres ved lavere temperaturer end resten.
For \ce{Ca(BH4)2} observeredes, udover rotationsdynamikken, en uidentificeret hændelse, som ifølge beregningerne sandsynligvis skyldes \ce{H2}-interstitielle defekter.
I god overensstemmelse med eksperimenter, aktiveres $C_3$-type rotationer ved lavere temperaturer end $C_2$-type rotationer.

For at undersøge reaktionsvejes omgivelser, blev der udviklet en metode til at finde højderyggen mellem førsteordens saddelpunkter på en multidimensional overflade.
Information om højderyggen kan bruges til at verificere den harmoniske approksimation brugt i forbindelse med transition-state-teorien anvendt til bestemmelse af reaktionshastigheder, specielt i forbindelse med verificeringen af at andenordens saddelpunkter, maksima langs højderyggen, er beliggende højere sammenlignet med førsteordens saddelpunkter.
%(I am not sure whether to use "ryg" or "højderyg" - the last one is used in geology and for describing mountains, whereas the first also simple means back (human's). Is multidimensional surface an OK word, it is only a surface if in 2d?. I do not think we have a Danish word for tst.)
Yderligere kan korrektioner til den harmoniske approksimation estimeres ved direkte evaluering af det konfigurative integerale langs højderyggen. (I am not sure about "konfigurative" have never heard about it..)
Nye minima langs højderyggen kan også identificeres sideløbende med reaktionsvejs-optimeringen, og derigennem kan yderligere transitions-mekanismer afsløres.
Metoden er baseret på modifikation af gradienten, for et sæt af punkter langs reaktionsvejen der forbinder saddelpunkterne, så den iterativt konvergerer mod højderyggen.
Ved hver iteration under optimeringen, inverteres gradienten langs en ustabil egentilstand vinkelret på reaktionsvejen. Dette afbilder effektivt højderyggen mod en minimum reaktionsvej, som kan findes med flere forskellige kendte teknikker.
Metoden blev anvendt til at studere Al adatom diffusion på Al(100) overfladen, for at finde højderyggen mellem en samlet bevægelse af 2, 3 eller 4 atomer, samt hoppe-mekanismer i forbindelse med diffusion.
Der var signifikante korrektioner for samlet bevægelse af 3 og 4 atomer. 
%(I am not sure if adatom is OK in Danish...)
Højderyg-metoden er en enkel metode som kan anvendes til at verificere reaktionshastigheder, men har potentialet for at bestemme mere nøjagtige hastigheder i sig selv, ved at repræsentere transition-state med højderyggen.



%\selectlanguage{english}
