\chapter*{Abstract}
\addcontentsline{toc}{chapter}{Abstract}

Reaction rates in theoretical chemistry was the main focus of this thesis.
Applying current methods to aid analysis of experimental data in complex borohydrides and improving these commonly used methods.

Complex borohydrides are materials of high hydrogen storage capacity and too high thermodynamic stability.
Understanding the high stability is of great importance to the development of suitable materials for hydrogen storage.
In an effort to gain insight into the structural transitions of two such materials, \ce{Ca(BH4)2} and \ce{Mg(BH4)2}, low temperature rotational dynamics experiments were performed.
The work presented in here revolved around assisting in the data analysis by performing density functional theory calculations on the possible dynamical events.
For the \ce{Mg(BH4)2}, only rotational dynamics were detected and they were in good agreement with the experimental values, showing that $C_2$-type rotations happen at lower temperatures than $C_3$-type rotations and that approximately $15\%$ of the \ce{BH4} units activate at a lower temperature than the rest.
For the \ce{Ca(BH4)2}, in addition to the rotational dynamics, longer range diffusion was detected.
Most likely this was due to \ce{H2}-interstitial defects.
The rotational dynamics were more prominent in the data and good agreement with the experiments was reached, showing that $C_3$-type rotations activate at lower temperatures than $C_2$-type rotations.

A method for finding the ridge between first order saddle points on a multidimensional surface was developed.
For atomic scale systems, such points on the energy surface correspond to atomic rearrangement mechanisms.
Information about the ridge can be used to test the validity of the harmonic approximation to transition state theory for reaction rates,
in particular to verify that second order saddle points - maxima along the ridge - are high enough compared to the first order saddle points.
Furthermore, corrections to the harmonic approximation can be estimated by direct evaluation of the configuration integral along the ridge.
New minima along the ridge can also be identified during the path optimisation,
thereby revealing additional transition mechanisms.
The method is based on modifying the gradient of a set of points along a path connecting the saddle points to iteratively converge to the ridge.
%The method is based on optimising a string of discretisation points along a path between the saddle points and using an iterative optimisation which requires only the force acting on the atoms.
At each iteration during the optimisation, the gradient is inverted along an unstable eigenmode perpendicular to the path, effectively mapping the ridge to a minimum energy path.
The method was applied to Al adatom diffusion on the Al(100) surface to find the ridge between 2-, 3- and 4-atom concerted displacement
and hop mechanisms for diffusion.
