\chapter*{Abstract}
\addcontentsline{toc}{chapter}{Abstract}

Finding the mechanisms and estimating the rate of chemical reactions is an essential part of modern research of atomic scale systems.
In this thesis, the application of well established methods to important systems for hydrogen storage is considered before developing extensions to further identify the reaction environment for a more accurate rate.

Complex borohydrides are materials of high hydrogen storage capacity and high thermodynamic stability (too high for hydrogen storage).
%Understanding the high stability is of great importance to the development of suitable materials for hydrogen storage.
In an effort to gain insight into the structural transitions of two such materials, \ce{Ca(BH4)2} and \ce{Mg(BH4)2}, experiments on low temperature rotational dynamics were performed.
The work presented here revolved around assisting in the data analysis by performing density functional theory calculations on the possible dynamical events.
For the \ce{Mg(BH4)2}, in good agreement with the experiments, $C_2$-type rotations occur at lower temperature than $C_3$-type rotations and approximately $15\%$ of the \ce{BH4} units activate at a lower temperature than the rest.
For the \ce{Ca(BH4)2}, in addition to the rotational dynamics, an unidentified event was detected which, according to the calculations was most likely due to \ce{H2}-interstitial defects.
In good agreement with the experiments, $C_3$-type rotations activate at lower temperature than $C_2$-type rotations.

In order to investigate the environment of reaction pathways, a method for finding the ridge between first order saddle points on a multidimensional surface was developed.
%For atomic scale systems, such points on the energy surface correspond to atomic rearrangement mechanisms.
Information about the ridge can be used to test the validity of the harmonic approximation to transition state theory for reaction rates,
in particular to verify that second order saddle points - maxima along the ridge - are high enough compared to the first order saddle points.
Furthermore, corrections to the harmonic approximation can be estimated by direct evaluation of the configuration integral along the ridge.
New minima along the ridge can also be identified during the path optimisation,
thereby revealing additional transition mechanisms.
The method is based on modifying the gradient of a set of points along a path connecting the saddle points to iteratively converge to the ridge.
%The method is based on optimising a string of discretisation points along a path between the saddle points and using an iterative optimisation which requires only the force acting on the atoms.
At each iteration during the optimisation, the gradient is inverted along an unstable eigenmode perpendicular to the path, locally mapping the ridge to a minimum energy path which can be located using various techniques.
The method was applied to Al adatom diffusion on the Al(100) surface to find the ridge between 2-, 3- and 4-atom concerted displacement
and hop mechanisms for diffusion.
Significant corrections were offered for the 3- and 4-atom concerted displacements.
The method offers a simple-to-use way to check the validity of reaction rates but has the potential to offer more accurate rates on its own by representing the transition state with the ridge.
